%% mascots.tex

\documentclass[conference]{IEEEtran}
% If IEEE MASCOTS is part of the IEEE Computer Society- range of
% conferences, be sure to add the compsoc option to the document class.
% If IEEEtran.cls has not been installed into the LaTeX system files,
% manually specify the path to it like:
%\documentclass[conference]{../sty/IEEEtran}

\usepackage{hyperref}

% correct bad hyphenation here
\hyphenation{op-tical net-works semi-conduc-tor}

% These should probably be merged into the same bibliography list in
% the end, but it might be good to have them seperately now so that
% it's easier to filter out possibly unreliable sources.
\newcommand{\dvtcmdcitebib}[2][]{\cite{#2}} % TODO: \masccitebib
\newcommand{\dvtcmdciteref}[2][]{\cite{#2}} % TODO: \mascciteref
%\newcommand{\dvtcmdcitefur}[2][]{\citefur{#2}}

\begin{document}

% Placeholder title:
\title{%
  Accelerating Graphics in\\
  the Simics Full-system Simulator}

\author{\IEEEauthorblockN{Eric Nilsson}
\IEEEauthorblockA{Intel Corporation\\
Email: \href{mailto:eric.nilsson@intel.com}{eric.nilsson@intel.com}}}

\maketitle

%% abstract.tex

\begin{abstract}

Full-system simulators provide benefits to developers in terms of a more rapid development cycle; since development may begin prior to that of next-generation hardware being available.
However, there is a distinct lack of graphics virtualization in industry-grade virtual platforms, leading to performance issues that may obfuscate the benefits virtual platforms otherwise have over execution on actual hardware.

This paper concerns the implementation of graphics acceleration by the means of paravirtualizing OpenGL~ES~$2.0$ in the Simics full-system simulator.
The study illustrates the benefits and drawbacks of paravirtualized methodology, in addition to performance analysis and comparison with the Android emulator; which likewise utilize paravirtualization to accelerate simulated graphics.

In this paper, we propose a solution for paravirtualized graphics using magic instructions; the implementation of which is subsequently described.
Additionally, three benchmarks are devised to stress key points in the developed solution; comprising areas such as inter-system communication latency and bandwidth.
Furthermore, the solution is evaluated based on computationally intensive applications.

For the purpose of this study, elapsed frame times for respective benchmarks are collected and compared with four platforms; i.e. the hardware accelerated host machine, the paravirtualized Android emulator, the software rasterized Simics - and the paravirtualized Simics platforms.

This paper establishes paravirtualization as a feasible method to achieve accelerated graphics in virtual platforms; accelerating graphics up to $34$ times of that of software rasterized counterparts.
Furthermore, the study concludes magic instructions as the primary bottleneck of communication latency in the devised solution.

\end{abstract}

% introduction.tex

% Introduction
\section{Introduction}
\label{sec:introduction}
Virtual platforms are becoming an important tool in the software industry in order to provide cost-effective time-to-market gains and meet the ever-shortening product life-cycles~\cite{journals:magnusson:2002, journals:yi:2006, publications:leupers:2010, publications:aarno:2014}.
Virtual platforms deliver these time-to-market benefits in two major ways.
First of all, virtual platforms enable pre-silicon development; that is, software development that may begin prior to next-generation hardware being available~\masccite[p.~52]{journals:magnusson:2002, publications:aarno:2014}.
Secondly, virtual platforms may provide additional development tools compared to working with actual hardware.
For example, some virtual platforms allow simulated systems, often known as simulation targets, to be stopped synchronously without affecting timing or state of the target software~\masccite[p.~61]{inproceedings:yu:2012, inproceedings_albertsson:2000}, and allow investigation of race conditions and other parallel programming issues~\masccite[p.~1]{inproceedings:schumacher:2010, publications:leupers:2010, journals:blum:2013}.
Additionally, such platforms may allow intricate inspection of simulated hardware, such as memories, caches, and registers~\masccite[p.~54]{journals:magnusson:2002}.
Some virtual platforms provide advanced features such as reverse execution (the ability to run a simulation backwards) and checkpointing (functionality to save and restore the state of a simulation)~\masccite{publications:aarno:2014, journals:aarno:2013}.
These features are useful for debugging and testing a diverse range of software, from firmware to end-user applications~\masccite[p.~25]{publications:leupers:2010}.

Common approaches to accelerate simulation performance include creating a functionally accurate model of the GPU, where internal details may be simplified, or using software rasterization without involving the GPU model.
However, these methodologies traditionally incur heavy performance losses.
Alternatively, one may delegate GPU workloads to the system on which the simulation runs (often referred to as the simulation host).
There are a number of ways to do so, such as relying on PCI~passthrough and similar technologies to grant access to underlying host hardware from within the virtual platform~\masccite[p.~415,~416]{inproceedings:regola:2010}, or utilizing a concept commonly referred to as "paravirtualization" at a higher level of abstraction, e.g., the graphics library.
Paravirtualization is defined as selectively modifying the virtual architecture to enhance scalability, performance, or simplicity~\masccite[p.~165-166]{inproceedings:youseff:2006}.
Effectively, this entails modifying the virtual machine to be similar, but not identical, to host hardware~\masccite[p.~165]{journals:barham:2003}.
As such, one may simplify the virtualization process by neglecting some hardware compatibility~\masccite[p.~1]{inproceedings:youseff:2006}.

This paper comprises an investigation into OpenGL graphics paravirtualization in the Simics full-system simulator.
The work presents an implementation of accelerated OpenGL~ES~$2.0$ graphics using magic instructions as a communications bridge between target and host systems.
Said implementation is evaluated using performance benchmarks stressing important attributes of the devised solution, and subsequently compared to regular software rasterization on the simulated platform.
Furthermore, the study identifies performance bottlenecks that may obstruct paravirtualized real-time graphics.
The results presented in this paper show performance improvements of up to $34$ times compared to software rasterized counterparts.


% For peer review papers, you can put extra information on the cover
% page as needed:
% \ifCLASSOPTIONpeerreview
% \begin{center} \bfseries EDICS Category: 3-BBND \end{center}
% \fi
%
% For peerreview papers, this IEEEtran command inserts a page break and
% creates the second title. It will be ignored for other modes.
\IEEEpeerreviewmaketitle

\section*{Acknowledgment}
\ldots

\bibliographystyle{IEEEtran}
\bibliography{mascots}

\end{document}

% Notes:
% no keywords
% conference papers do not normally have an appendix

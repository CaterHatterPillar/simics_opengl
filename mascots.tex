%% mascots.tex

% TODO Major:
% * Make it apperent that the paper studies the performance of accelerated
%   graphics on a platform using virtualization hardware extensions to speed up
%   execution on x86. Accentuate why it is probable that x86 is not the target
%   platform, and that the solution would probably speed up graphics by orders
%   of magnitude more should virtualization extensions not be used.
% * Briefly mention that OpenGL ES 2.0 is chosen because it's often used for
%   embedded platforms (which is one of Simics' strenths).
% * The section on GPU Modeling can probably be shortened quite a bit. Consider
%   breaking out the contents of the subsections under the overhead title.
% * Consider placing the section on OpenGL Framework Generation after
%   descriptions of target and host libraries.
% * I've commented out descriptions on the setup of the QEMU Android
%   emulator from the Experimental Methodology section. Consider
%   adding them to the document if we're to expand on the comparisons
%   with the android emulator.
% * The Platform Comparison section (commented out under Methods and Results)
%   should probably be uncommented and moved to a later section if we want to
%   bring up QEMU.
% * Explain in detail why paravirtualization is a good balancing of
%   the problems facing graphics acceleration in system simulation.
% * Refer to the benchmarks, not at Chess or Julia, but as the 'latency' and
%   'compute'/'performance' (?) benchmarks.
% * The benchmarks are open source. Reference the repository.
% * Explain Future Work?
% * Merge the Target and Host system libraries sections.
% * Merge the Simics Pipe and page table traversal sections.
% * The 'read-first-line-from-file'-command seems to carry about some
%   unnecessary whitespace with it. Remove. When this has been done, it is
%   plausible that the tables may be put side-by-side.

% TODO Minor:
% * Correct formatting of multiple references [4][34] -> [4,34].
% * Indentation at the start of sections.

% Thesis chapters yet to import contents from:
% * Discussion (?)
% * Future work (consider adding some future work bits)

\documentclass[conference]{IEEEtran}
% If IEEE MASCOTS is part of the IEEE Computer Society- range of
% conferences, be sure to add the compsoc option to the document class.
% If IEEEtran.cls has not been installed into the LaTeX system files,
% manually specify the path to it like:
%\documentclass[conference]{../sty/IEEEtran}

\usepackage{soul} % Used to temporarily highlight areas of text during writing.
\usepackage{xcolor} % colorbox for the inline code command.
\usepackage{hyperref} % Hyperlinks and urls.
\usepackage{listings} % code boxes.
\usepackage{amsfonts} % \circledR.
\usepackage{graphicx} % Displaying pdf graphics.
\usepackage{multirow} % Used for specialized tables (multirow).
\usepackage{rotating} % Used to rotate figures sideways.
\usepackage{todonotes} % Used to add TODOs in the document. Remove when no longer needed.
\usepackage[mediumspace,mediumqspace,squaren,binary]{SIunits} % \milli\second

% correct bad hyphenation here
\hyphenation{op-tical net-works semi-conduc-tor}

% These should probably be merged into the same bibliography list in
% the end, but it might be good to have them seperately now so that
% it's easier to filter out possibly unreliable sources.  When most
% contents have been imported from the thesis, replace these with
% suitable commands.
\newcommand{\dvtcmdcitebib}[2][]{\cite{#2}} % TODO: \masccitebib
\newcommand{\dvtcmdciteref}[2][]{\cite{#2}} % TODO: \mascciteref
%\newcommand{\dvtcmdcitefur}[2][]{\citefur{#2}}
\newcommand{\dvtcmdrefname}[1]{\nameref{#1}}

% Formatting inline code command:
\newcommand{\dvtcmdcodeinline}[1]{\colorbox{black!5}{\lstinline[basicstyle=\ttfamily\color{black}]{#1}}}
% Command to read, and ouput the first line of a file:
\newread\file
\newcommand{\dvtcmdfirstline}[1]{\input{#1}\unskip}

\begin{document}

\title{%
  Accelerating Graphics in\\
  the Simics Full-system Simulator}

\author{\IEEEauthorblockN{Eric Nilsson\IEEEauthorrefmark{1},
    Daniel Aarno\IEEEauthorrefmark{2}, and Erik Carstensen\IEEEauthorrefmark{3}}
  \IEEEauthorblockA{Intel Corporation\\
    Stockholm, Sweden\\
    Email: \IEEEauthorrefmark{1}\href{mailto:eric.nilsson@intel.com}{eric.nilsson@intel.com},
    \IEEEauthorrefmark{2}\href{mailto:daniel.aarno@intel.com}{daniel.aarno@intel.com},
    \IEEEauthorrefmark{3}\href{mailto:erik.carstensen@intel.com}{erik.carstensen@intel.com}}
  \and
  \IEEEauthorblockN{Prof. H{\aa}kan Grahn}
  \IEEEauthorblockA{Blekinge Institute of Technology\\
    Karlskrona, Sweden\\
    Email: \href{mailto:hakan.grahn@bth.se}{hakan.grahn@bth.se}}}

\maketitle

%% abstract.tex

\begin{abstract}

Full-system simulators provide benefits to developers in terms of a more rapid development cycle; since development may begin prior to that of next-generation hardware being available.
However, there is a distinct lack of graphics virtualization in industry-grade virtual platforms, leading to performance issues that may obfuscate the benefits virtual platforms otherwise have over execution on actual hardware.

This paper concerns the implementation of graphics acceleration by the means of paravirtualizing OpenGL~ES~$2.0$ in the Simics full-system simulator.
The study illustrates the benefits and drawbacks of paravirtualized methodology, in addition to performance analysis and comparison with the Android emulator; which likewise utilize paravirtualization to accelerate simulated graphics.

In this paper, we propose a solution for paravirtualized graphics using magic instructions; the implementation of which is subsequently described.
Additionally, three benchmarks are devised to stress key points in the developed solution; comprising areas such as inter-system communication latency and bandwidth.
Furthermore, the solution is evaluated based on computationally intensive applications.

For the purpose of this study, elapsed frame times for respective benchmarks are collected and compared with four platforms; i.e. the hardware accelerated host machine, the paravirtualized Android emulator, the software rasterized Simics - and the paravirtualized Simics platforms.

This paper establishes paravirtualization as a feasible method to achieve accelerated graphics in virtual platforms; accelerating graphics up to $34$ times of that of software rasterized counterparts.
Furthermore, the study concludes magic instructions as the primary bottleneck of communication latency in the devised solution.

\end{abstract}

% introduction.tex

% Introduction
\section{Introduction}
\label{sec:introduction}
Virtual platforms are becoming an important tool in the software industry in order to provide cost-effective time-to-market gains and meet the ever-shortening product life-cycles~\cite{journals:magnusson:2002, journals:yi:2006, publications:leupers:2010, publications:aarno:2014}.
Virtual platforms deliver these time-to-market benefits in two major ways.
First of all, virtual platforms enable pre-silicon development; that is, software development that may begin prior to next-generation hardware being available~\masccite[p.~52]{journals:magnusson:2002, publications:aarno:2014}.
Secondly, virtual platforms may provide additional development tools compared to working with actual hardware.
For example, some virtual platforms allow simulated systems, often known as simulation targets, to be stopped synchronously without affecting timing or state of the target software~\masccite[p.~61]{inproceedings:yu:2012, inproceedings_albertsson:2000}, and allow investigation of race conditions and other parallel programming issues~\masccite[p.~1]{inproceedings:schumacher:2010, publications:leupers:2010, journals:blum:2013}.
Additionally, such platforms may allow intricate inspection of simulated hardware, such as memories, caches, and registers~\masccite[p.~54]{journals:magnusson:2002}.
Some virtual platforms provide advanced features such as reverse execution (the ability to run a simulation backwards) and checkpointing (functionality to save and restore the state of a simulation)~\masccite{publications:aarno:2014, journals:aarno:2013}.
These features are useful for debugging and testing a diverse range of software, from firmware to end-user applications~\masccite[p.~25]{publications:leupers:2010}.

Common approaches to accelerate simulation performance include creating a functionally accurate model of the GPU, where internal details may be simplified, or using software rasterization without involving the GPU model.
However, these methodologies traditionally incur heavy performance losses.
Alternatively, one may delegate GPU workloads to the system on which the simulation runs (often referred to as the simulation host).
There are a number of ways to do so, such as relying on PCI~passthrough and similar technologies to grant access to underlying host hardware from within the virtual platform~\masccite[p.~415,~416]{inproceedings:regola:2010}, or utilizing a concept commonly referred to as "paravirtualization" at a higher level of abstraction, e.g., the graphics library.
Paravirtualization is defined as selectively modifying the virtual architecture to enhance scalability, performance, or simplicity~\masccite[p.~165-166]{inproceedings:youseff:2006}.
Effectively, this entails modifying the virtual machine to be similar, but not identical, to host hardware~\masccite[p.~165]{journals:barham:2003}.
As such, one may simplify the virtualization process by neglecting some hardware compatibility~\masccite[p.~1]{inproceedings:youseff:2006}.

This paper comprises an investigation into OpenGL graphics paravirtualization in the Simics full-system simulator.
The work presents an implementation of accelerated OpenGL~ES~$2.0$ graphics using magic instructions as a communications bridge between target and host systems.
Said implementation is evaluated using performance benchmarks stressing important attributes of the devised solution, and subsequently compared to regular software rasterization on the simulated platform.
Furthermore, the study identifies performance bottlenecks that may obstruct paravirtualized real-time graphics.
The results presented in this paper show performance improvements of up to $34$ times compared to software rasterized counterparts.

% previousresearch.tex

% Previous Research
\section{Previous Research}
\label{sec:previousresearch}

System simulators are abundant and exist in corporate~\dvtcmdcitebib{magazines:bohrer:2004}, academic~\dvtcmdcitebib{journals:rosenblum:1995}, and open-source variations~\dvtcmdciteref{magazines:bartholomew:2006}.
Such platforms, like Simics, have been used for a variety of purposes including, but not limited to, thermal control strategies in multicores~\dvtcmdcitebib{inproceedings:bartolini:2010}, networking timing analysis~\dvtcmdcitebib{journals:ortiz:2009}, web server performance evaluation~\dvtcmdcitebib{journals:villa:2005}, and to simulate costly hardware financially unfeasible to researchers~\dvtcmdcitebib{journals:alameldeen:2003}.
Furthermore, such simulators may also be used to port OSs to new processors~\dvtcmdciteref{technicaldocs:netbsd:2014}.

For the purposes of graphics acceleration, possible strategies are numerous.
From a number of core strategies, such as device modeling, passthrough technologies, and paravirtualization, there are many attempts at effective GPU virtualization; many of which require modification of both target- and host systems (such as the development of specialized passthrough drivers~\dvtcmdcitebib{inproceedings:lagarcavilla:2007}).
One such instance is presented by Hansen in his work on the Blink display system~\dvtcmdcitebib{inproceedings:hansen:2007}.
Neither are the concepts of advanced simulatory features new to GPU virtualization, as there are multiple attempts to implement the checkpoint/restart model in a GPU context, such as the work by Guo et al. concerning the CUDA framework~\dvtcmdcitebib{inproceedings:guo:2013}.
Another solution that also supports the checkpoint/restart scheme is VMGL by Lagar-Cavilla et al., which by the means of paravirtualization accelerates the OpenGL~$1.5$ framework~\dvtcmdcitebib{inproceedings:lagarcavilla:2007}.
The groundwork produced by Lagar-Cavilla et al. showcases the potential for paravirtualized graphics since the VMGL framework, for a certain set of benchmarks, attains improvements of roughly two orders of magnitude in comparison to software rasterization.

Current promising GPU virtualization projects include the Virgil3D-project~\dvtcmdciteref{technicaldocs:qemudevel:2014}.
As described at the project homepage, the project strives to create a virtual GPU which may subsequently utilize host hardware to accelerate 3D rendering.
The project is currently being maintained, again according to the project's GitHub homepage, by Red~Hat's Dave Airlie.
Other related works include modeling GPU devices in the QEMU full-system simulator with software OpenGL~ES rasterization support, as presented by Shen et al.~\dvtcmdcitebib{inproceedings:shen:2010}.

At the time of writing, graphics virtualization is no longer limited to the academic community but is also existent in the industry, as big virtualization players incorporate various graphics acceleration solutions in their products.
One such example is VMware, Inc.~\dvtcmdciteref{technicaldocs:vmware:2014}.

% thesissimics.tex

% Simics
\subsection{Simics}
\label{sec:simics}
Simics is a full-system simulator developed by Intel\circledR and sold through Intel\circledR subsidiary Wind River Systems, Inc.
Simics was originally developed by the simulation group at the Swedish Institute of Computer Science \footnote{This being the first instance of an academic group running an unmodified OS in an entirely simulated environment}; the members of which founded Virtutech \footnote{Virtutech was acquired by Intel\circledR in $2010$~\dvtcmdciteref{web:miller:2010}.} and commercially launched the product in $1998$~\dvtcmdcitebib{journals:magnusson:2013}.

As an architectural simulator, Simics primary client group is software- and systems developers that produce an assortment of software for complex systems involving software and hardware interaction~\dvtcmdcitebib{journals:aarno:2013}.
As such, key attributes of Simics are scalability, repeatability, and high-performance simulation.
For these purposes, the simulator utilizes hardware-assisted virtualization, and other performance boosting technologies such as hypersimulation ~\dvtcmdcitebib[p.~38]{publications:leupers:2010}.
Simics also features a number of advanced functionalities, adhering to the deterministic nature of the simulator, such as checkpointing (see section \dvtcmdrefname{sec:simics_checkpointing}) and reverse execution (see section \dvtcmdrefname{sec:simics_reverseexecution})~\dvtcmdcitebib{publications:leupers:2010}.

The ability to simulate the entirety of an unmodified software stack has led to Simics being used to simulate a variety of systems including, but not limited to, single-processor embedded boards, multiprocessor servers, and heterogeneous telecom clusters~\dvtcmdcitebib{journals:aarno:2013}.
Current employers of the Simics full-system simulator include IBM ~\dvtcmdcitebib[p.~12:1,~12:6]{journals:koerner:2009}, NASA ~\dvtcmdciteref{web:windriver:2014}\dvtcmdciteref{web:nasa:2014}, and Intel\circledR ~\dvtcmdcitebib[p.~100]{journals:veselyi:2013}.
Other past and current employers of the simulator include Sun Microsystems, Inc., Ericsson, and Hewlett-Packard Company ~\dvtcmdcitebib{journals:magnusson:2013}, in addition to Cisco Systems, Inc., Freescale Semiconductor, Inc., GE Aviation, Honeywell International, Inc., Lockheed Martin, Nortel Networks Corporation and Northrop Grumman Corporation ~\dvtcmdciteref{web:miller:2010}.
Additionally, the simulator has a strong academic tradition; being known to operate in over $300$ universities throughout the world~\dvtcmdcitebib[p.~252]{journals:villa:2005}.

% Deterministic Execution
\subsubsection{Deterministic Execution}
\label{sec:simics_deterministicexecution}
'Deterministic execution' commonly refers to the execution of deterministic algorithms; meaning that a certain function, given a definite input, will produce a decisive output - throughout the process in which the system passes through a distinct set of states (see \dvtcmdcitebib{journals:cohen:1979} for an overview on deterministic and non-deterministic algorithms).
Some sources have voiced concerns over the consequences, in terms of debugging, of non-deterministic behavior caused by concurrent software~\dvtcmdcitebib[p.~3-5]{journals:lee:2006}\dvtcmdcitebib[p.~92]{journals:holzmann:1997}.
Some even go as far as to argue that determinism is a prerequisite for effective debugging and testing~\dvtcmdcitebib[p.~3,~4]{dissertation:devietti:2012}\dvtcmdcitebib[p.~51,~59]{inproceedings:yu:2012}.
As such, deterministic behavior may be seen as a valuable attribute in a virtual platform (see \dvtcmdcitebib[p.~1,~2]{papers:bergan:2011} for an overview on the value of determinism in software development).

In Simics, 'deterministic execution ' denotes the entire target system as a deterministic algorithm, wherein all instructions are executed in a deterministic manner on the simulated hardware~\dvtcmdcitebib[p.~20,~21]{journals:aarno:2013}.
This means that the simulated system (e.g., an OS ), presuming the same input, will allocate the same memory space in virtual memories, receive the same number of interrupts in sequence, and even inhabit the same registers in virtual CPUs~\dvtcmdcitebib[p.~19,~20]{journals:aarno:2013} (see \dvtcmdciteref{web:engblom:2012:determinism} for a brief rundown of deterministic execution in Simics ).
As such, given an arbitrary number of instructions, the simulation state may be recreated indiscriminately down to the level of the instruction set and corresponding cycles.
Thus, throughout this document and in relation to the Simics full-system simulator, deterministic execution refers to the deterministic state transition of a target system; as described in this section.

% Checkpointing
\subsection{Checkpointing}
\label{sec:simics_checkpointing}
The state of a computer system may be defined as the entirety of its stored information, or memory, at a given time~\dvtcmdcitebib[p.~103]{publications:harris:2007}.
It may be beneficial to store such states in a 'checkpoint/restart ' scheme, as suggested by Jiang et al. in regards to CUDA kernels~\dvtcmdcitebib[p.~196-197,~210]{inproceedings:zhang:2013}.
In this way, developers may save- and restore the state of a system which can reduce overhead of restarting computationally heavy applications from scratch~\dvtcmdcitebib[p.~19,~20]{journals:aarno:2013}.

In Simics, 'checkpointing ' refers to the functionality to save the complete state of a simulation into a portable format.
When applied, this format is known as a checkpoint.
This ability not only  saves time in terms of program initialization and debugging~\dvtcmdcitebib[p.~54]{journals:magnusson:2002}, but may also ease testing and collaboration in-between developers as checkpoints may be distributed and shared~\dvtcmdcitebib[p.~20,~21]{journals:aarno:2013} (see \dvtcmdciteref{web:engblom:2013:collaboratingusingsimics} for an overview on checkpointing in Simics ).
As such, throughout this document and in relation to the Simics full-system simulator, checkpointing denotes this functionality.

% Reverse Execution
\subsection{Reverse Execution}
\label{sec:simics_reverseexecution}
'Reverse execution' provides software developers with the ability to return, often from portable checkpoints, to previous states of execution~\dvtcmdcitebib[p.~2,~3]{journals:akgul:2004}.
This may be useful when debugging, profiling, or testing as difficult-to-reach system states may be stored and returned to, effectively bypassing program initialization and other hindrances~\dvtcmdcitebib[p.~54]{journals:magnusson:2002}.

In Simics, 'reverse execution ' denotes said ability - spanning the entirety of the simulated system~\dvtcmdcitebib[p.~30,~31]{publications:leupers:2010}.
As such, simulated systems may be run in reverse; including virtualized hardware device states, disk contents and CPU registers~\dvtcmdcitebib[p.~20,~21]{journals:aarno:2013} (see \dvtcmdciteref{web:engblom:2013:backtorevexe} for an overview on reverse execution in Simics ); whilst still maintaining determinism in the simulated system.
As such, throughout this document and in relation to the Simics full-system simulator, reverse execution refers to the functionality described in this section.

\subsection{OpenGL ES}
\label{sec:backgroundandrelatedwork_opengles}
OpenGL~ES is an API for 3D-graphics on an assortment of embedded systems, such as mobile devices or vehicle displays~\dvtcmdcitebib{publications:munshi:2008}.
The OpenGL~ES set of APIs is developed by Khronos, the same consortium responsible for the development of the OpenGL APIs, which - unlike OpenGL for embedded systems - is intended for desktop graphics~\dvtcmdcitebib{publications:wright:2010}.
The OpenGL~ES APIs are traditionally derived from the standard OpenGL APIs, and are thus similar in appearance albeit more limited~\dvtcmdcitebib{publications:wright:2010}.

The OpenGL~ES~$2.0$ API is selected for use due to its programmable shaders and reduced function set, in addition to accelerated support for the API in the Android emulator.

%% The OpenGL~ES specification is intended to adhere to the following specifications, in relation to the original OpenGL APIs~\dvtcmdcitebib{publications:munshi:2008}:
%% \begin{itemize}[noitemsep]
%%         \item Reduce complexity, but attempt maintain compatibility with OpenGL when possible.
%%         \item Optimize power consumption for embedded devices.
%%         \item Incorporate a set of quality specifiers into the OpenGL~ES specification. This includes minimum quality specifiers for image quality into the standard; accommodating for limited screen sizes in embedded systems.
%% \end{itemize}

%% OpenGL~ES~$2.0$ consists of two Khronos specifications; being the OpenGL~ES~$2.0$ API specification and the OpenGL~ES Shading Language Specification~\dvtcmdcitebib{publications:munshi:2008}.
%% Being derived from the OpenGL ~$2.0$ specification~\dvtcmdcitebib{publications:wright:2010}, OpenGL~ES~$2.0$ supports programmable shaders and is no longer encumbered by the fixed functionality that characterized earlier versions of the OpenGL and OpenGL~ES APIs~\dvtcmdcitebib{publications:munshi:2008}; making the API a modern contestant amongst a variety of applications on a range of platforms, such as the Android and iOS operating systems~\dvtcmdcitebib{publications:wright:2010}, popular on modern smartphones.

\subsection{GPU Architecture}
\label{sec:backgroundandrelatedwork_gpuarchitecture}
GPUs are massively parallel numeric computing processors often used for 3D graphics acceleration.
While CPUs are designed to maximize sequential performance, GPUs are designed to maximize floating-point operations per second throughput; originally required for the floating point linear algebra often needed by 3D graphics.
Additionally, the continuous need for faster processing units, slowed down by the heat dissipation issues limiting clock frequency since 2003~\dvtcmdcitebib{publications:kirk:2010}, has driven the utilization of GPUs for general purpose workloads.

Generally, CPUs are designed to optimize the execution of an individual thread.
In order to facilitate sequential performance, the processing pipeline is designed for low-latency operations; including thorough caching methodologies and low-latency arithmetic units.
This design philosophy is often referred to as latency-oriented design, since it strives for low-latency operations to accommodate high performance of individual threads~\dvtcmdcitebib{publications:kirk:2010}.

Meanwhile, GPUs are designed to maximize the the throughput of a large number of threads; having less regard for the performance of individual threads.
As such, GPUs do not prioritize caches or other low-latency optimizations like CPUs, but aim to conceal overhead induced by memory references or arithmetic by the execution of many threads, which may perform in place during arithmetic or memory operations.
Such design is often referred to as throughput-oriented design~\dvtcmdcitebib{publications:kirk:2010}.

The differing design philosophies of CPU and GPU units are so fundamentally different that they are largely incompatible in terms their workloads, as explained by Kirk and Hwu~\dvtcmdcitebib{publications:kirk:2010}.
As such, programs that feature few threads, CPUs with lower operation latencies may perform better than GPUs; whereas a program with a large number of threads may fully utilize the higher execution throughput of GPUs~\dvtcmdcitebib{publications:kirk:2010}.

The introduction of a number of CPU-oriented optimizations for simulating GPU-bound workloads on CPUs has made it possible to simulate such kernels several times faster than traditional simulation.
Although orders of magnitudes slower, such optimizations may ease the simulation of GPU workloads on CPUs, as indicative by some studies~\dvtcmdcitebib{papers:nilsson:2013}.
However, without hardware-assisted virtualization, the execution of GPU workloads on CPUs quickly becomes unfeasible; inducing the need of more advanced graphics simulation methodologies.

% Graphics Virtualization
\subsection{Graphics Virtualization}
\label{sec:backgroundandrelatedwork_graphicsvirtualization}
There are a number of ways of virtualizing GPUs in system simulators, a few of which accommodate for hardware acceleration of GPU kernels.
When faced with tackling the issue of GPU virtualization, there are equally many variables to consider as there are options; the first of which is the purpose of said virtualization.
The Simics architectural simulator is by all means a full-system simulator; meaning, as portrayed in chapter \ref{sec:simics}, that it may run real-software stacks without modification.
However, Simics is intended to feature low level timing fidelity for the purposes of high performance, and is - as such - not a cycle-accurate simulator.
In this way, and in line with the considerations for GPU virtualization, one must analyze and balance the purpose of simulation since there is not always a general best-case.
As such, methodologies with varying levels of simulatory accuracy present themselves - from slow low-level instruction set modeling to fast high level paravirtualization of an assortment of graphics frameworks.
Summaries of these methodologies are presented in this section.

% GPU Modeling
\paragraph{GPU Modeling}
\label{par:backgroundandrelatedwork_graphicsvirtualization_gpumodeling}
One may consider developing a full-fletched GPU model; that is, virtualizing the GPU ISA.
This methodology may be appropriate for the purposes of low-level development close to GPU hardware.
For example, one might imagine the scenario of driver development for next-generation GPUs.

However, the development of GPU models, similar to that of common architectural model development for the Simics full-system simulator, incurs a number of flaws.
The first of these flaws encompasses estimated development costs reaching unsustainable levels, due to GPU hardware often being poorly documented~\dvtcmdcitebib{inproceedings:lagarcavilla:2007} on the contrary to CPU architectures.
Furthermore, the modeling of massively parallelized GPU technology on CPUs induce high costs rendering the methodology less preferable for development requiring high application speed.

% PCI Passthrough
\paragraph{PCI Passthrough}
\label{par:backgroundandrelatedwork_graphicsvirtualization_pcipassthrough}
Furthermore, one ought to examine the benefits of PCI~passthrough; allowing virtual systems first-hand access to host machine devices~\dvtcmdciteref{web:jones:2009}.
The direct contact with host system devices accommodated by methodologies such as PCI~passthrough enable fully-fledged hardware accelerated graphics.
Yet the methodology suffers from several disadvantages, such as requiring dedicated hardware, causing the host system to lose access to devices during the course of simulation.
In terms of GPU virtualization, this would induce the necessity of the host machine featuring multiple graphics cards.
Additionally, and mayhaps the biggest flaw of passthrough methodologies, is the requirement of modifying the simulation target to utilize host hardware; effectively restricting what systems may be simulated.
This restriction encompasses, inter alia, the utilization of corresponding device drivers to the host system, rendering the methodology inflexible in terms of GPU virtualization diversity.
As such, and in line with a paravirtualized approach, PCI~passthrough requires modification of the target system - in addition to configuration of the simulation host.

% Soft Modeling
\paragraph{Soft Modeling}
\label{par:backgroundandrelatedwork_graphicsvirtualization_softmodeling}
As an alternative to precise modeling of GPU devices, one might analyze the feasibility of high-speed software rasterization.
Albeit not up to hardware accelerated speeds, some results indicate an increased feasibility of high-speed software rasterization in modern graphics frameworks (see \dvtcmdciteref{papers:nilsson:2013}), where traditional software rasterization is accelerated using thread pooling optimizations and SIMD technologies~\dvtcmdciteref{web:microsoft:2013:warp}; all for the purposes of optimizing execution for CPU, rather than GPU, execution.
As such, one may avoid some of the overhead induced by simulating GPU workload on CPUs, which is traditionally not fit for purpose.
One might speculate that using such technologies in collaboration with hardware-assisted virtualization might bring software rasterization up to competitive speeds fit for some simulatory development purposes, \hl{replacing} the need for more sophisticated virtualization techniques.
However, without native execution speeds or used with complex GPU workloads, the technique may fall short.

% Paravirtualization
\paragraph{Paravirtualization}
\label{par:backgroundandrelatedwork_graphicsvirtualization_paravirtualization}
At a higher level of abstraction, there is the option of virtualization by paravirtualization.
By selectively modifying target systems, one may modify the inner workings of system attributes and add functionality such as graphics hardware support.
For the purposes of graphics acceleration, such a system attribute may be a graphics library or a kernel driver.

Often, paravirtualized methodologies incurs the benefits of host hardware acceleration of some graphics framework, and is implemented at a relatively high abstraction level (see figure \ref{fig:overview}).
Inherent from its higher abstraction, paravirtualization may be relatively cost-effective to implement in comparison to alternatives such as GPU modeling.
Additionally, virtualizing at the graphics library software level circumvents the need for users to re-link or modify the application they wish accelerated.
Furthermore, the serialization of framework invocations by the means of fast communications channels may accommodate for significant performance improvements when compared to that of networking solutions (see section \ref{sec:proposedsolutionandimplementation_simicspipe}).

However, despite the possibility for significant performance improvements, graphics virtualization by the means of paravirtualization is not without inherent flaws.
In particular, a paravirtualized graphics library may be expensive to maintain as frameworks evolve and specifications change.
Additionally, the means of paravirtualization requires the target system to be modified; albeit not necessarily being a substantial defect as a paravirtualized framework may still accelerate unmodified target applications utilizing the library.
In this way, paravirtualization may be considered to be a decent leveling of the benefits and drawbacks of the various virtualization methodologies presented in this section.

% QEMU
\subsection{QEMU}
\label{sec:backgroundandrelatedwork_qemu}
QEMU ('Quick~Emulator') is an open-source virtual platform described as a full system emulator~\dvtcmdcitebib[p.~1]{inproceedings:bellard:2005} and a high-speed functional simulator~\dvtcmdcitebib[p.~1]{inproceedings:shen:2010} (see \dvtcmdciteref[p.~69]{magazines:bartholomew:2006} for an overview of QEMU).
It supports simulation of several common system architectures and hardware devices and can, like Simics, save and restore the state of a simulation~\dvtcmdcitebib[p.~1]{inproceedings:bellard:2005}.
As such, QEMU may, like Simics, run unmodified target software such as OSs, drivers, and other applications.
The platform is widely used in academia, and is the subject of several articles and reports cited throughout this document, such as the graphics acceleration described by Lagar-Cavilla et al.~\dvtcmdcitebib{inproceedings:lagarcavilla:2007}, and the work by Guo et al.~\dvtcmdcitebib{inproceedings:guo:2013}.
Additionally, QEMU powers the Android emulator, which helps mobile developers bring about software for the Android OS.

The Android emulator is described as a virtual mobile device emulator~\dvtcmdciteref{web:google:2013:usingtheemulator}.
Included in the Android SDK, it supports virtualization of an assortment of mobile hardware configurations.
In the presence of the Android $4.0.3$ release, the Android SDK was updated to make use of hardware-assisted x86 virtualization; significantly increasing the performance of CPU-bound workloads for x86 systems~\dvtcmdciteref{web:ducrohet:2012:afasteremulator}.
In addition to this, Google implemented hardware support for the OpenGL~ES $1.1$- and $2.0$ frameworks; granting developers utilizing said frameworks hardware acceleration of graphics~\dvtcmdciteref{web:ducrohet:2012:afasteremulator}.
Google's solution (see \dvtcmdciteref{technicaldocs:google:2014}), consists of a paravirtualized implementation which circumvents the simulation by forwarding its OpenGL~ES invocations to the host system by using networking sockets or directly via the simulator program (note, however, that there is no software rasterized solution for running OpenGL~ES~$2.0$ in the Android emulator).
As of Android $4.4$, the Android emulator uses QEMU to simulate ARM and x86 devices aiding those wishing to develop software for mobile units (by using images devised by Intel\circledR, the Android emulator may run the Android OS on x86 simulated hardware (see section \ref{sec:experimentalmethodology_platformconfiguration})).

% Magic Instructions
\subsection{Magic Instructions}
\label{sec:backgroundandrelatedwork_magicinstructions}
Sometimes during system simulation, there may be reasons as to why one would like to escape the simulation and resume execution in the real world.
Such a scenario would be a debugging breakpoint, to share data in-between target and host systems, or for any reason modify the simulation state.
There are a number of ways to communicate with the outside world (including the host machine) from within the simulation, such as by networking means or specially devised kernel drivers, but few are as instant as the - arguably - legitimately coined 'magic instruction'.

The magic instruction is a concept used to denote a \dvtcmdcodeinline{nop}-type instruction, meaning an instruction that would have no effect if run on the target architecture (such as \dvtcmdcodeinline{xchg ebx, ebx} - '\hl{Swap contents in registers} \dvtcmdcodeinline{ebx} \hl{and} \dvtcmdcodeinline{ebx}' - on the x86 -architecture), which, when executed on the simulated hardware in a virtual platform, invokes a callback-method in the simulation host~\dvtcmdcitebib[p.~32]{publications:leupers:2010}.
An advantage of this methodology is an often negligible invocation cost, as the context switch is often instant from the perspective of the target system~\dvtcmdcitebib[p.~131]{journals:rechistov:2013}.
Furthermore, being a greatly desirable attribute, magic instructions require no modification of the target system.
Another advantage of the magic instruction paradigm is that the system invoking such an instruction may, without complications, run outside of a simulation - as this would simply result in regular \dvtcmdcodeinline{nop}-behavior.
In effect, implementation of magic instructions requires replacing one- or more instructions in the target instruction set; thereby making the magic instruction platform-dependent.
However, the solution is often designed to only respond to magic instructions wherein a certain magic number, sometimes called a 'leaf number'~\dvtcmdcitebib[p.~131]{journals:rechistov:2013}, is present in an arbitrary processor register.

%% % Virtual Time
%% \subsection{Virtual Time}
%% \label{sec:backgroundandrelatedwork_virtualtime}
%% In terms of system simulation, time often becomes abstract; since it is not necessarily the same for an observer outside of the simulation as that of an observer from the inside.
%% The variance in virtual time, as compared to that of real-world time, is called 'simulation slowdown' and may reach orders of magnitude faster than that of real-world time, or likewise orders of magnitude slower.

%% The concepts of real-world and virtual time are particularly important when considering performance measurements.
%% When attempting to establish some sort of measurement in a full-system simulator, such as Simics, one must contemplate what type of time is relevant to the study being performed.
%% For graphics acceleration of real-time applications, it is likely that the real-world wall clock is the primary point of reference (see section \ref{sec:threatstovalidity_platformprofiling} for an elaboration on how time measurement is performed for the sake of this study).
%% However, there are cases in which virtual time is worthwhile to profile, such as the performance of virtual system drivers.

% problemformulation.tex

% Problem Formulation
\section{Problem Formulation}
\label{sec:problemformulation}
\hl{In regards to previous work in the area, there is no indication -- in academic writing -- of existing paravirtualized graphics in a simulator with advanced capabilities such as Simics, featuring deterministic execution, checkpointing and reverse execution.
Potential performance gains on such a platform are inherently unclear due to these features.}
Such functionality could simplify debugging, testing, and profiling of applications comprising some GPU-bound workload.
Entailed by these research gaps, the research questions formulated in this chapter are considered to be lacking in the field.\todo{Explain angle in regards to Lagar-Cavilla.}

This paper presents the paravirtualization of OpenGL~ES~$2.0$ in the Simics full-system simulator. 
The study concerns investigating the performance of paravirtualized graphics in a modern virtual platform.
This entails development and analysis of efficient communication and execution in the Simics run-time environment.
The objectives of the paper is to evaluate the feasibility of paravirtualization as an approach to accelerate graphics in virtual platforms, and to identify the strengths and weaknesses of using magic instructions as a communications bridge.

To accommodate the analysis of benefits and drawbacks of paravirtualized graphics, two benchmarks are devised.
The benchmarks are designed to stress latency in target-to-host communication and computational intensity brought on by complex GPU workloads.
As such, the purposes of these benchmarks is to stress a suspected bottleneck in the communication between target and host systems, and to measure how well the solution performs under computational stress.
The performance of these benchmarks is compared to that of traditional software rasterization.

Thus, this study is relevant to the field of computer science by expanding upon the the knowledge of graphics acceleration in virtual platforms.
Graphics acceleration in a virtual platform is relevant because it facilitates debugging, testing, and profiling of software which depends on GPU graphics acceleration.
By these means, this paper contributes to the field of computer science by answering these questions from the perspective of graphics paravirtualization in the Simics full-system simulator.
Formally, paper seeks to answer the following research questions:

\begin{enumerate}
  \item What constitutes a viable implementation of paravirtualized graphics?
  \item What are the benefits and disadvantages of paravirtualized graphics?
  \item How does paravirtualized graphics performance relate to software rasterization?
\end{enumerate}

% methodsandresults.tex

% Methods and Results
\section{Methods and Results}
\label{sec:methodsandresults}
OpenGL paravirtualization in Simics encompasses three overall components: the target system libraries, the host system libraries, and a communications channel between them named the "Simics pipe".
Most OpenGL glue code, on both target and host, is generated by a program from specification files detailing function signatures and arguments.
An exception is methods that require state saving, which are implemented manually.

The target system libraries implement the OpenGL and EGL (the interface between OpenGL and the underlying platform windowing system) APIs; unmodified binaries in the target system are subsequently linked with these libraries as with any other OpenGL or EGL implementation.
However, instead of communicating with the graphics device, the target system libraries serialize and forward the command stream to the simulation host.
The transmission is not necessarily performed at once, nor in the designated order, because of uncertainties regarding argument data proportion.
For instance, the number of vertices to be rendered does not have to be apparent at a given time, but implicit in a later OpenGL invocation.
Accordingly, certain function calls have to be delayed until more information is known about the OpenGL state.

In collaboration with the target system libraries, the host system libraries decode and interpret the received byte stream.
Subsequently, the host system libraries may safely perform the relayed workload and return any results to the target system.

Both target and host system libraries maintain a subset of the OpenGL state, such as bound vertex buffers and attribute properties.
These states must be maintained because of the asynchronous nature of the command stream.

Because of differences in the creation and maintenance of windows on different platforms (Fedora, Android, etc.), the window to which OpenGL renders is kept on the simulation host.
This is problematic; the target system libraries must communicate with a fraudulent window in the simulation host -- \textit{and} the native window.
For example, it is important that the native window reports successful initialization, lest the OpenGL application concludes an error and quits.
The issue is overcome by selectively overriding symbols in the target libraries so that a subset of functions may be overloaded.
This way, one may extend the original EGL library to invoke the simulation host prior to performing its actions.

To communicate with the simulation host, the Simics pipe uses "magic instructions".
A magic instruction is a \masccodeinline{nop} instruction that invokes a callback-method in the simulation host when executed on simulated hardware~\masccite[p.~32]{publications:leupers:2010}.
Because of the inherent performance demands brought on by real-time graphics, they constitute a suitable communications medium for rendering information between target and host systems.

During a magic instruction, we may utilize any available registers; the number and size of registers is the data-sharing bottleneck of this method.
Thus, we transmit the starting address of the serialized command stream in a 64-bit register.
Having escaped the simulation context, Simics can translate the transmitted virtual address to a physical one using the virtual machine MMU.
Consequently, the physical address can be used to locate the memory page in the simulated RAM image.
To ensure the pages are not swapped to disk when the simulation state is paused, we "lock" all pages constituting the command stream buffer.
Subsequently, all memory pages are continiously retrieved by iterating the original virtual address with the target page size, effectively traversing the virtual memory table.

\subsection{Experimental Methodology}
\label{sec:experimentalmethodology}
To evaluate the implementation, performance of paravirtualized graphics in Simics is compared to software rasterization.
Simics itself simulates an Intel\circledR ~Core\texttrademark ~i7 processor and an Intel\circledR ~X58 chipset.
Throughout simulation, hardware-assisted virtualization using KVM runs x86 instructions natively on the host hardware.
Like the host system, the simulation target runs Fedora~$19$ Linux and use the Mesa llvmpipe driver software rasterizer~\masccite{web:mesa:2015}.

The experiments are performed on a system with the following specifications:
\begin{itemize}
\item Intel\circledR\ Core\texttrademark\ i7-4770HQ
\item Intel\circledR\ Iris\texttrademark\ Pro Graphics 5200
\end{itemize}

Two benchmarks are devised on-site to stress suspected bottlenecks: one benchmark performs a large number of OpenGL invocations, while the other has a computationally intensive workload.
Given a target frame time of $16$~\milli\second , the benchmarks are configured to run at $10$ to $20$~\milli\second\ per frame, when hardware accelerated on the host system; a $16$~\milli\second\ frame time roughly corresponds to $60$~frames per second.
The benchmarks are shaped this way to reflect the expected load of a real-time interactive application.
As such, the benchmarks should be representative of typical scenarios induced by modern applications using OpenGL, such as responsive UIs.
The developed benchmark suite is open source~\masccite{web:intel:2014}.

For each benchmark, the elapsed times of $1000$ frames are collected.
To gain some understanding on how well the given performance scales, three instances of each benchmark are run with smaller and larger input data, tuned to yield approximately half and double frame time.
The specifics of each benchmark are described below.

% tables.tex

\providecommand{\chesskeyone}{$60\times60$ tiles}
\providecommand{\chesskeytwo}{$84\times84$ tiles}
\providecommand{\chesskeythree}{$118\times118$ tiles}

\providecommand{\juliakeyone}{$225$ iterations}
\providecommand{\juliakeytwo}{$450$ iterations}
\providecommand{\juliakeythree}{$900$ iterations}

\begin{table*}
  \parbox{.5\textwidth}{
    % tab:keyvalsimics
    \centering
    \tabcolsep=0.11cm % reduce column width
    \begin{tabular}{|c|c|c|c|c|c|}
      \hline
      \multirow{2}{*}{Benchmark} & \multirow{2}{*}{Input} & \multicolumn{4}{p{4cm}|}{\centering Elapsed time (\milli\second )} \\
      \cline{3-6} && \multicolumn{1}{c|}{Min} & \multicolumn{1}{c|}{Max} & \multicolumn{1}{c|}{Std} & \multicolumn{1}{c|}{Avg} \\ \hline
      \multirow{3}{*}{Chess} & \chesskeyone & \mascfirstline{simicschess60x60.dat.min} & \mascfirstline{simicschess60x60.dat.max}	& \mascfirstline{simicschess60x60.dat.std} & \mascfirstline{simicschess60x60.dat.avg} \\ %\cline{2-6}
      & \chesskeytwo & \mascfirstline{simicschess84x84.dat.min} & \mascfirstline{simicschess84x84.dat.max} & \mascfirstline{simicschess84x84.dat.std} & \mascfirstline{simicschess84x84.dat.avg} \\ %\cline{2-6}
      & \chesskeythree & \mascfirstline{simicschess118x118.dat.min} & \mascfirstline{simicschess118x118.dat.max} & \mascfirstline{simicschess118x118.dat.std} & \mascfirstline{simicschess118x118.dat.avg} \\ \hline
      \multirow{3}{*}{Julia} & \juliakeyone & \mascfirstline{simicsjulia225.dat.min} & \mascfirstline{simicsjulia225.dat.max} & \mascfirstline{simicsjulia225.dat.std} & \mascfirstline{simicsjulia225.dat.avg} \\ %\cline{2-6}
      & \juliakeytwo & \mascfirstline{simicsjulia450.dat.min} & \mascfirstline{simicsjulia450.dat.max} & \mascfirstline{simicsjulia450.dat.std} & \mascfirstline{simicsjulia450.dat.avg} \\ %\cline{2-6}
      & \juliakeythree & \mascfirstline{simicsjulia900.dat.min} & \mascfirstline{simicsjulia900.dat.max} & \mascfirstline{simicsjulia900.dat.std} & \mascfirstline{simicsjulia900.dat.avg} \\ \hline
    \end{tabular}
    \vspace{5pt}
    \caption{Software rasterization results in Simics.}
    \label{tab:keyvalsimics}
  }
  \hfill
  \parbox{.5\textwidth}{
    % tab:keyvalpara
    \centering
    \begin{tabular}{|c|c|c|c|c|c|}
      \hline
      \multirow{2}{*}{Benchmark} & \multirow{2}{*}{Input} & \multicolumn{4}{p{4cm}|}{\centering Elapsed time (\milli\second )} \\
      \cline{3-6} && \multicolumn{1}{c|}{Min} & \multicolumn{1}{c|}{Max} & \multicolumn{1}{c|}{Std} & \multicolumn{1}{c|}{Avg} \\ \hline
      \multirow{3}{*}{Chess} & \chesskeyone & \mascfirstline{parachess60x60.dat.min} & \mascfirstline{parachess60x60.dat.max} & \mascfirstline{parachess60x60.dat.std} & \mascfirstline{parachess60x60.dat.avg} \\
      & \chesskeytwo & \mascfirstline{parachess84x84.dat.min} & \mascfirstline{parachess84x84.dat.max} & \mascfirstline{parachess84x84.dat.std} & \mascfirstline{parachess84x84.dat.avg} \\
      & \chesskeythree & \mascfirstline{parachess118x118.dat.min} & \mascfirstline{parachess118x118.dat.max} & \mascfirstline{parachess118x118.dat.std} & \mascfirstline{parachess118x118.dat.avg} \\ \hline
      \multirow{3}{*}{Julia} & \juliakeyone & \mascfirstline{parajulia225.dat.min} & \mascfirstline{parajulia225.dat.max}	& \mascfirstline{parajulia225.dat.std} & \mascfirstline{parajulia225.dat.avg} \\
      & \juliakeytwo & \mascfirstline{parajulia450.dat.min} & \mascfirstline{parajulia450.dat.max} & \mascfirstline{parajulia450.dat.std} & \mascfirstline{parajulia450.dat.avg} \\
      & \juliakeythree & \mascfirstline{parajulia900.dat.min} & \mascfirstline{parajulia900.dat.max} & \mascfirstline{parajulia900.dat.std} & \mascfirstline{parajulia900.dat.avg} \\ \hline
    \end{tabular}
    \vspace{5pt}
    \caption{Paravirtualization results in Simics.}
    \label{tab:keyvalpara}
  }
\end{table*}


% Benchmark: Chess
\paragraph{Benchmark: Chess}
\label{par:experimentalmethodology_benchmarking_benchmarkchess}
To stress the latency between target and host systems, the 'Chess' benchmark performs a multitude of lightweight OpenGL invocations per frame, rendering a grid of chess-like tiles.
For each frame rendered, depending on the number of tiles, the benchmark performs a large number of magic instructions.
This induces high utilization of the Simics Pipe, which is intended to stress suspected magic instruction overhead.

A long sequence of draw calls is representative of drawing multitudes of shapes with OpenGL, such as a UI.
Accordingly, the benchmark is suitable for the purpose of representing a large number of graphics invocations.

The Chess benchmark is run with $60\times60$, $84\times84$, and $118\times118$ tiles, which entails $9\times60\times60$, $9\times84\times84$, and $9\times118\times118$ magic instructions per frame.

% Benchmark: Julia
\paragraph{Benchmark: Julia}
\label{par:experimentalmethodology_benchmarking_benchmarkjulia}
To stress the computational prowess of paravirtualized graphics in Simics, the 'Julia' benchmark performs a lone, computationally intensive, OpenGL invocation that render the Julia fractal~\masccite{web:tsiombikas:2014}.
Like a benchmark that renders a grid of tiles, where the grid resolution may be adjusted, the computation of a fractal is trivially scalable in terms of complexity, by modifying the number of iterations the computing algorithm performs.
Therefore, the Julia benchmark is suitable to profile a computationally intensive workload.

The Julia benchmark is run with $225$, $450$, and $900$ iterations, all of which induce $16$ magic instructions per frame.

\subsection{Threats to Validity}
\label{sec:threatstovalidity}
Because of complications caused by virtual time, measuring time in system simulation sometimes dictate special measures.
For instance, in terms of real-time rendering, the observer is far more interested in a frame rate relative to wall-clock time rather than virtual time.

In order to measure frame time in relation to wall-clock time, profiling must take place outside of the simulation.
One way of achieving this is to listen in on activity passing through a target serial port; this is a traditional front-end to the machine.
In this way, a simulation breakpoint can be triggered at the occurrence of a certain sequence of bytes written to a UART serial port.
This is the method used to measure frame time in Simics.

When using serial ports in this manner, one may introduce a profiling cost.
For example, file descriptors do not immediately transmit a byte sequence over the system UART.
For our set-up, we measure this overhead cost to be, on average, $1.5$~\milli\second .
If not specified otherwise, presented results take into account this average.

\subsection{Results}
\label{sec:results}
Results accumulated from software rasterized and paravirtualized execution are presented in Tables~\ref{tab:keyvalsimics}~and~\ref{tab:keyvalpara}.
In Figure~\ref{fig:histogram}, the results are presented as histograms, visualizing elapsed time in milliseconds to sample density.
For each experiment, collected frame time samples ($1000$) are subdivided into $100$ bins.
Any measurements outside of the standard deviation are not included in the figures.

Figure~\ref{fig:histogram} indicates that the Chess benchmark, when software rasterized, yields a broad sample distribution, seemingly distributed around a single point.
The right-hand side of the graph, showing impaired performance induced by paravirtualization, visualize a distribution decrease.
This is supported by the data presented in Table~\ref{tab:keyvalpara}.
We observe that software rasterization outperforms its paravirtualized counterpart, regardless of the number of tiles rendered.
The benchmark is devised to identify any bottlenecks related to the number of paravirtualizaed invocations.
Evidently, the prediction of a target-to-host communication latency issue is confirmed.

In Simics, magic instructions incur a context switch cost when resuming execution on the host.
This causes the simulation to no longer execute natively, which inhibits performance imrovements granted by hardware-assisted virtualization.
It also entails that Simics can no longer utilize JIT, forcing the simulator to rely on interpretation.
As such, in great numbers, magic instructions may affect performance.

By measuring the elapsed time of using magic insructions to escape simulation $1000$ times, we conclude that $1000$ consecutive magic instructions induce an average overhead of $5$~\milli\second\ (not taking into account any profiling cost).
These findings indicate that magic instruction overhead could account for the majority of elapsed frame time.

Figure~\ref{fig:histogram} shows that the Julia benchmark yields double to triple peak sample density destribution, both in software rasterized and paravirtualed Simics.
What causes this behavior is unclear; the fractal algorithm should perform roughly equally frame-to-frame.
The benchmark is intented to demonstrate how paravirtualization performs under computational stress.
This is where the benefits of hardware acceleration should be made apperent.
Accordingly, weaknesses in software rasterization is highlighted with frame times above the two second mark; corresponding maximum paravirtualized frame time measuring a mere $156$ \milli\second.
This confirms performance improvements of using paravirtualization, paired with magic instructions, to accelerate graphics.

These measurements are collected using hardware-assisted virtualization for accelerated virtualization.
If hardware-assisted virtualization is not available, such as if the simulated platform is other than x86, we expect a major hit to performance.
For software rasterization, this impact accounts for frame time increases well over two orders of magnitude.
Meanwhile, performance impacts to paravirtualization is often not significant, sometimes as low as a third of the original frame time.
For the Chess benchmark, paravirualized frame time increase with \textit{up to} one order of magnitude, one order less than the performance hit to software rasterization.
Across the board, paravirtualization suffer less performance impact, rendering the benchmarks up to three orders of magnitude faster than software rasterization.
We conclude that the effects of paravirtualization increase by one order of magnitude without hardware-assisted virtualization.
This entails that workloads that are otherwise sub-optimal for paravirtualization -- those performing a large amount of function invocations -- bring about performance improvements when utilizing paravirtualization.
Thus, the impact of magic instruction overhead is reduced, likely because a costly context switch is not inflicted on Simics.
We reason that some software rasterized worklads (Chess) may attain decent simulation performance simply because of a fast simulator; when native execution is not available, neither JIT nor interpretation may attain the same speeds as paravirtualization.

The samples collected without hardware-assisted virtualization are not presented in detail, since the scenario of native execution is far more likely.

% conclusion.tex

% Conclusion
\section{Conclusion}
\label{sec:conclusion}
In section \ref{sec:results}, we have established strengths and weaknesses of paravirtualized graphics in the Simics full-system simulator; most notably, the bottleneck introduced by magic instruction overhead.
As such, we have confirmed original suspicions through the use of our benchmarks.
Thus, our study has identified the performance bottleneck inherent in great numbers of paravirtualized function invocations for magic instructions.

Furthermore, compiled results have showcased great improvements for computationally intensive graphics, as demonstrated by the Julia fractal benchmark, compared to its software rasterized Simics counterpart.
As such, we have accelerated graphics by up to $34$ times, reducing frame time from that of $1415$~\milli\second\ to the real-time feasible count of 42~\milli\second .
For platforms that cannot utilize hardware-assisted virtualization, paravirtualization has been estimated to obtain an additional order of magnitude in frame time reduction in comparison to software rasterization.
Such scenarios may include simulating other platforms than x86, such as PowerPC, or when utilizing certain advanced features in Simics -- e.g. breakpoints.

Accordingly, our experiment has identified the potential of using paravirtualization with shared memory for the means of accelerating graphics to that of real-time performance, testimonial to the results presented by Lagar-Cavilla et al. in their work on using paravirtualization to accelerate graphics~\dvtcmdcitebib{inproceedings:lagarcavilla:2007}.
Additionally, beyond that of accelerated graphics, results indicate performance improvements in terms of maximum frame times, inducing significantly improved standard deviation.
In line with stable frame rates being prerequisites for real-time applications, this further indicates, in coagency with reduced frame times, the feasibility of utilizing paravirtualized methodology for the purposes of accelerating graphics within virtual platforms.

To summarize: this paper has presented a solution for graphics acceleration implemented in the Simics full-system simulator by the means of paravirtualization.
The end-result is a solution which may generate libraries imitating EGL and OpenGL libraries.
This solution may effectively spy on application EGL utilization, without inhibiting said exchange, allowing unmodified OpenGL applications to be accelerated from within the simulation target.
The implementation communicates by the means of low-latency magic instructions with no limit as to how much memory may be shared.
As such, throughout this document, we have tackled and presented several issues pertaining to paravirtualized graphics acceleration.
For the purposes of performance testing, we have developed benchmarks with the distinct purpose of highlighting solution weaknesses and strengths.
We have presented an analysis of benchmarking results and concluded the benefits and drawbacks of paravirtualization as means to graphics acceleration in virtual platforms, backed by hard data stressing key points in the implementation.
Accordingly, the findings of this paper has contributed to our understanding of the difficulties facing paravirtualized graphics acceleration, and has established the feasibility of using paravirtualization to accelerate graphics in virtual platforms to that of real-time qualities.

To conclude: This paper has demonstrated performance improvements by accelerating graphics using paravirtualization.
Induced benefits are performance improvements of up to $34$ times -- or two orders of magnitude, paired with larger benefits in non-hardware-assisted virtualized use-cases.
Magic instruction overhead has been identified as the main performance bottleneck.
As such, a drawback of graphics paravirtualization is a weakness to large amounts of framework invocations.
Thus, this paper claims paravirtualization as a successful formula for system simulator graphics acceleration, and suggests utilizing high-level paravirtualization to accelerate graphics in virtual platforms.

The presented implementation may be advanced in a number of ways in order to support a higher number of platforms and an array of performance enhancements.
Below, recommendations for future study are presented.

In terms of performance, command serialization batching should be considered for the purposes of minimizing the number of performed magic instructions.
That is, the ability to queue framework invocations and transmit them in a batch rather than individually.
Considering that magic instructions are a performance bottleneck, such an optimization could drastically improve simulation performance.

Often, it is desirable to simulate systems other than the simulation host.
One could pose the scenario of a Linux host system simulating a machine running a Windows OS.
In this case, it is possible that target software utilize the DirectX framework to render graphics, whereas the host Linux system only feature the OpenGL libraries.

In $2014$, Valve released software capable of converting DirectX~$9.0$c-code to that of OpenGL~\dvtcmdciteref{technicaldocs:valve:2014}.
Albeit limited in its capabilities, the functionality of translating between one, platform-specific, framework to that of a cross-platform one may be practical for the purposes of graphics acceleration in virtual platforms.
An inherent flaw in paravirtualizing graphics APIs is that a target application may only make use of accelerated simulation performance if that application is implemented using the API in question.
If such a solution could be used to translate the rendering of a target program on-the-fly in a virtual machine, this might bridge the gap between the performance gains of using paravirtualization and other -- albeit slower -- more general ways of modeling graphics for any type of system.
Thus, the adaption of this sort of software could extend the capabilities and area-of-application of graphics paravirtualization in the Simics full-system simulator.

% Other possible future work to consider:
% * More verbose benchmarks.
% * Memory bandwidth benchmark (see EOF).
% * Consider mentioning making further study into how the overhead cost
%   for magic instructions might change if using VMP, JIT, or regular
%   interpretation.

%% Furthermore, and for the purposes of complementing this dissertation in particular, the author would like to suggest additional tests stressing target -to-host communications.
%% Preferably, said tests would stress the communication by other means than profiling the sampling of a large texture, since such a test may cause volatile performance in the reference material (being software rasterized Simics ), possibly due to cache misses (see section \ref{sec:threatstovalidity_benchmarkvariations}).
%% When performing such a test, it may be of value to profile the overhead induced by the memory table traversal described in section \ref{sec:proposedsolutionandimplementation_pagetabletraversal}.


% For peer review papers, you can put extra information on the cover
% page as needed:
% \ifCLASSOPTIONpeerreview
% \begin{center} \bfseries EDICS Category: 3-BBND \end{center}
% \fi
%
% For peerreview papers, this IEEEtran command inserts a page break and
% creates the second title. It will be ignored for other modes.
\IEEEpeerreviewmaketitle

\section*{Acknowledgment}
\ldots

\bibliographystyle{IEEEtran}
\bibliography{mascots}

\end{document}

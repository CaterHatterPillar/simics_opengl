%% mascots.tex

\documentclass[conference]{IEEEtran}
% If IEEE MASCOTS is part of the IEEE Computer Society- range of
% conferences, be sure to add the compsoc option to the document class.
% If IEEEtran.cls has not been installed into the LaTeX system files,
% manually specify the path to it like:
%\documentclass[conference]{../sty/IEEEtran}

\usepackage{hyperref}

% correct bad hyphenation here
\hyphenation{op-tical net-works semi-conduc-tor}

% These should probably be merged into the same bibliography list in
% the end, but it might be good to have them seperately now so that
% it's easier to filter out possibly unreliable sources.
\newcommand{\dvtcmdcitebib}[2][]{\cite{#2}} % TODO: \masccitebib
\newcommand{\dvtcmdciteref}[2][]{\cite{#2}} % TODO: \mascciteref
%\newcommand{\dvtcmdcitefur}[2][]{\citefur{#2}}

\begin{document}

% Placeholder title:
\title{%
  Accelerating Graphics in\\
  the Simics Full-system Simulator}

\author{\IEEEauthorblockN{Eric Nilsson}
\IEEEauthorblockA{Intel Corporation\\
Email: \href{mailto:eric.nilsson@intel.com}{eric.nilsson@intel.com}}}

\maketitle

%% abstract.tex

\begin{abstract}

Full-system simulators provide benefits to developers in terms of a more rapid development cycle; since development may begin prior to that of next-generation hardware being available.
However, there is a distinct lack of graphics virtualization in industry-grade virtual platforms, leading to performance issues that may obfuscate the benefits virtual platforms otherwise have over execution on actual hardware.

This paper concerns the implementation of graphics acceleration by the means of paravirtualizing OpenGL~ES~$2.0$ in the Simics full-system simulator.
The study illustrates the benefits and drawbacks of paravirtualized methodology, in addition to performance analysis and comparison with the Android emulator; which likewise utilize paravirtualization to accelerate simulated graphics.

In this paper, we propose a solution for paravirtualized graphics using magic instructions; the implementation of which is subsequently described.
Additionally, three benchmarks are devised to stress key points in the developed solution; comprising areas such as inter-system communication latency and bandwidth.
Furthermore, the solution is evaluated based on computationally intensive applications.

For the purpose of this study, elapsed frame times for respective benchmarks are collected and compared with four platforms; i.e. the hardware accelerated host machine, the paravirtualized Android emulator, the software rasterized Simics - and the paravirtualized Simics platforms.

This paper establishes paravirtualization as a feasible method to achieve accelerated graphics in virtual platforms; accelerating graphics up to $34$ times of that of software rasterized counterparts.
Furthermore, the study concludes magic instructions as the primary bottleneck of communication latency in the devised solution.

\end{abstract}


% For peer review papers, you can put extra information on the cover
% page as needed:
% \ifCLASSOPTIONpeerreview
% \begin{center} \bfseries EDICS Category: 3-BBND \end{center}
% \fi
%
% For peerreview papers, this IEEEtran command inserts a page break and
% creates the second title. It will be ignored for other modes.
\IEEEpeerreviewmaketitle

\section{Introduction}
Citation\dvtcmdcitebib{journals:lee:2006}\dvtcmdciteref{web:microsoft:2013:warp}.

\section*{Acknowledgment}
\ldots

\bibliographystyle{IEEEtran}
\bibliography{mascots}

\end{document}

% Notes:
% no keywords
% conference papers do not normally have an appendix

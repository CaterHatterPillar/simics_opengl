%% simics_opengl.tex

\documentclass[conference, compsoc]{IEEEtran}

\usepackage{xcolor} % colorbox for the inline code command.
\usepackage{hyperref} % Hyperlinks and urls.
\usepackage{listings} % code boxes.
\usepackage{amsfonts} % \circledR.
\usepackage{graphicx} % Displaying pdf graphics.
\usepackage{multirow} % Used for specialized tables (multirow).
\usepackage{enumitem} % Compact lists.
\usepackage[mediumspace,mediumqspace,squaren,binary]{SIunits} % \milli\second

% correct bad hyphenation here
\hyphenation{op-tical net-works semi-conduc-tor power-pc}

% Compact lists (see enumitem):
\setitemize{noitemsep,topsep=0pt,parsep=0pt,partopsep=0pt}

\newcommand{\masccite}[2][]{\cite{#2}}
\newcommand{\masccodeinline}[1]{\colorbox{black!5}{\lstinline[basicstyle=\ttfamily\color{black}]{#1}}}
\newread\file
\newcommand{\mascfirstline}[1]{\input{#1}\unskip}

\begin{document}

\title{Accelerating Graphics in the Simics Full-system Simulator}

\author{\IEEEauthorblockN{Eric Nilsson\IEEEauthorrefmark{1},
    Daniel Aarno\IEEEauthorrefmark{2}, and Erik Carstensen\IEEEauthorrefmark{3}}
  \IEEEauthorblockA{Intel Corporation\\
    Stockholm, Sweden\\
    Email: \IEEEauthorrefmark{1}\href{mailto:eric.nilsson@intel.com}{eric.nilsson@intel.com},
    \IEEEauthorrefmark{2}\href{mailto:daniel.aarno@intel.com}{daniel.aarno@intel.com},
    \IEEEauthorrefmark{3}\href{mailto:erik.carstensen@intel.com}{erik.carstensen@intel.com}}
  \and
  \IEEEauthorblockN{H{\aa}kan Grahn}
  \IEEEauthorblockA{Blekinge Institute of Technology\\
    Karlskrona, Sweden\\
    Email: \href{mailto:hakan.grahn@bth.se}{hakan.grahn@bth.se}}}

\maketitle

%% abstract.tex

\begin{abstract}

Full-system simulators provide benefits to developers in terms of a more rapid development cycle; since development may begin prior to that of next-generation hardware being available.
However, there is a distinct lack of graphics virtualization in industry-grade virtual platforms, leading to performance issues that may obfuscate the benefits virtual platforms otherwise have over execution on actual hardware.

This paper concerns the implementation of graphics acceleration by the means of paravirtualizing OpenGL~ES~$2.0$ in the Simics full-system simulator.
The study illustrates the benefits and drawbacks of paravirtualized methodology, in addition to performance analysis and comparison with the Android emulator; which likewise utilize paravirtualization to accelerate simulated graphics.

In this paper, we propose a solution for paravirtualized graphics using magic instructions; the implementation of which is subsequently described.
Additionally, three benchmarks are devised to stress key points in the developed solution; comprising areas such as inter-system communication latency and bandwidth.
Furthermore, the solution is evaluated based on computationally intensive applications.

For the purpose of this study, elapsed frame times for respective benchmarks are collected and compared with four platforms; i.e. the hardware accelerated host machine, the paravirtualized Android emulator, the software rasterized Simics - and the paravirtualized Simics platforms.

This paper establishes paravirtualization as a feasible method to achieve accelerated graphics in virtual platforms; accelerating graphics up to $34$ times of that of software rasterized counterparts.
Furthermore, the study concludes magic instructions as the primary bottleneck of communication latency in the devised solution.

\end{abstract}

% introduction.tex

% Introduction
\section{Introduction}
\label{sec:introduction}
Virtual platforms are becoming an important tool in the software industry in order to provide cost-effective time-to-market gains and meet the ever-shortening product life-cycles~\cite{journals:magnusson:2002, journals:yi:2006, publications:leupers:2010, publications:aarno:2014}.
Virtual platforms deliver these time-to-market benefits in two major ways.
First of all, virtual platforms enable pre-silicon development; that is, software development that may begin prior to next-generation hardware being available~\masccite[p.~52]{journals:magnusson:2002, publications:aarno:2014}.
Secondly, virtual platforms may provide additional development tools compared to working with actual hardware.
For example, some virtual platforms allow simulated systems, often known as simulation targets, to be stopped synchronously without affecting timing or state of the target software~\masccite[p.~61]{inproceedings:yu:2012, inproceedings_albertsson:2000}, and allow investigation of race conditions and other parallel programming issues~\masccite[p.~1]{inproceedings:schumacher:2010, publications:leupers:2010, journals:blum:2013}.
Additionally, such platforms may allow intricate inspection of simulated hardware, such as memories, caches, and registers~\masccite[p.~54]{journals:magnusson:2002}.
Some virtual platforms provide advanced features such as reverse execution (the ability to run a simulation backwards) and checkpointing (functionality to save and restore the state of a simulation)~\masccite{publications:aarno:2014, journals:aarno:2013}.
These features are useful for debugging and testing a diverse range of software, from firmware to end-user applications~\masccite[p.~25]{publications:leupers:2010}.

Common approaches to accelerate simulation performance include creating a functionally accurate model of the GPU, where internal details may be simplified, or using software rasterization without involving the GPU model.
However, these methodologies traditionally incur heavy performance losses.
Alternatively, one may delegate GPU workloads to the system on which the simulation runs (often referred to as the simulation host).
There are a number of ways to do so, such as relying on PCI~passthrough and similar technologies to grant access to underlying host hardware from within the virtual platform~\masccite[p.~415,~416]{inproceedings:regola:2010}, or utilizing a concept commonly referred to as "paravirtualization" at a higher level of abstraction, e.g., the graphics library.
Paravirtualization is defined as selectively modifying the virtual architecture to enhance scalability, performance, or simplicity~\masccite[p.~165-166]{inproceedings:youseff:2006}.
Effectively, this entails modifying the virtual machine to be similar, but not identical, to host hardware~\masccite[p.~165]{journals:barham:2003}.
As such, one may simplify the virtualization process by neglecting some hardware compatibility~\masccite[p.~1]{inproceedings:youseff:2006}.

This paper comprises an investigation into OpenGL graphics paravirtualization in the Simics full-system simulator.
The work presents an implementation of accelerated OpenGL~ES~$2.0$ graphics using magic instructions as a communications bridge between target and host systems.
Said implementation is evaluated using performance benchmarks stressing important attributes of the devised solution, and subsequently compared to regular software rasterization on the simulated platform.
Furthermore, the study identifies performance bottlenecks that may obstruct paravirtualized real-time graphics.
The results presented in this paper show performance improvements of up to $34$ times compared to software rasterized counterparts.

% methodsandresults.tex

% Methods and Results
\section{Methods and Results}
\label{sec:methodsandresults}
OpenGL paravirtualization in Simics encompasses three overall components: the target system libraries, the host system libraries, and a communications channel between them named the "Simics pipe".
Most OpenGL glue code, on both target and host, is generated by a program from specification files detailing function signatures and arguments.
An exception is methods that require state saving, which are implemented manually.

The target system libraries implement the OpenGL and EGL (the interface between OpenGL and the underlying platform windowing system) APIs; unmodified binaries in the target system are subsequently linked with these libraries as with any other OpenGL or EGL implementation.
However, instead of communicating with the graphics device, the target system libraries serialize and forward the command stream to the simulation host.
The transmission is not necessarily performed at once, nor in the designated order, because of uncertainties regarding argument data proportion.
For instance, the number of vertices to be rendered does not have to be apparent at a given time, but implicit in a later OpenGL invocation.
Accordingly, certain function calls have to be delayed until more information is known about the OpenGL state.

In collaboration with the target system libraries, the host system libraries decode and interpret the received byte stream.
Subsequently, the host system libraries may safely perform the relayed workload and return any results to the target system.

Both target and host system libraries maintain a subset of the OpenGL state, such as bound vertex buffers and attribute properties.
These states must be maintained because of the asynchronous nature of the command stream.

Because of differences in the creation and maintenance of windows on different platforms (Fedora, Android, etc.), the window to which OpenGL renders is kept on the simulation host.
This is problematic; the target system libraries must communicate with a fraudulent window in the simulation host -- \textit{and} the native window.
For example, it is important that the native window reports successful initialization, lest the OpenGL application concludes an error and quits.
The issue is overcome by selectively overriding symbols in the target libraries so that a subset of functions may be overloaded.
This way, one may extend the original EGL library to invoke the simulation host prior to performing its actions.

To communicate with the simulation host, the Simics pipe uses "magic instructions".
A magic instruction is a \masccodeinline{nop} instruction that invokes a callback-method in the simulation host when executed on simulated hardware~\masccite[p.~32]{publications:leupers:2010}.
Because of the inherent performance demands brought on by real-time graphics, they constitute a suitable communications medium for rendering information between target and host systems.

During a magic instruction, we may utilize any available registers; the number and size of registers is the data-sharing bottleneck of this method.
Thus, we transmit the starting address of the serialized command stream in a 64-bit register.
Having escaped the simulation context, Simics can translate the transmitted virtual address to a physical one using the virtual machine MMU.
Consequently, the physical address can be used to locate the memory page in the simulated RAM image.
To ensure the pages are not swapped to disk when the simulation state is paused, we "lock" all pages constituting the command stream buffer.
Subsequently, all memory pages are continiously retrieved by iterating the original virtual address with the target page size, effectively traversing the virtual memory table.

\subsection{Experimental Methodology}
\label{sec:experimentalmethodology}
To evaluate the implementation, performance of paravirtualized graphics in Simics is compared to software rasterization.
Simics itself simulates an Intel\circledR ~Core\texttrademark ~i7 processor and an Intel\circledR ~X58 chipset.
Throughout simulation, hardware-assisted virtualization using KVM runs x86 instructions natively on the host hardware.
Like the host system, the simulation target runs Fedora~$19$ Linux and use the Mesa llvmpipe driver software rasterizer~\masccite{web:mesa:2015}.

The experiments are performed on a system with the following specifications:
\begin{itemize}
\item Intel\circledR\ Core\texttrademark\ i7-4770HQ
\item Intel\circledR\ Iris\texttrademark\ Pro Graphics 5200
\end{itemize}

Two benchmarks are devised on-site to stress suspected bottlenecks: one benchmark performs a large number of OpenGL invocations, while the other has a computationally intensive workload.
Given a target frame time of $16$~\milli\second , the benchmarks are configured to run at $10$ to $20$~\milli\second\ per frame, when hardware accelerated on the host system; a $16$~\milli\second\ frame time roughly corresponds to $60$~frames per second.
The benchmarks are shaped this way to reflect the expected load of a real-time interactive application.
As such, the benchmarks should be representative of typical scenarios induced by modern applications using OpenGL, such as responsive UIs.
The developed benchmark suite is open source~\masccite{web:intel:2014}.

For each benchmark, the elapsed times of $1000$ frames are collected.
To gain some understanding on how well the given performance scales, three instances of each benchmark are run with smaller and larger input data, tuned to yield approximately half and double frame time.
The specifics of each benchmark are described below.

% tables.tex

\providecommand{\chesskeyone}{$60\times60$ tiles}
\providecommand{\chesskeytwo}{$84\times84$ tiles}
\providecommand{\chesskeythree}{$118\times118$ tiles}

\providecommand{\juliakeyone}{$225$ iterations}
\providecommand{\juliakeytwo}{$450$ iterations}
\providecommand{\juliakeythree}{$900$ iterations}

\begin{table*}
  \parbox{.5\textwidth}{
    % tab:keyvalsimics
    \centering
    \tabcolsep=0.11cm % reduce column width
    \begin{tabular}{|c|c|c|c|c|c|}
      \hline
      \multirow{2}{*}{Benchmark} & \multirow{2}{*}{Input} & \multicolumn{4}{p{4cm}|}{\centering Elapsed time (\milli\second )} \\
      \cline{3-6} && \multicolumn{1}{c|}{Min} & \multicolumn{1}{c|}{Max} & \multicolumn{1}{c|}{Std} & \multicolumn{1}{c|}{Avg} \\ \hline
      \multirow{3}{*}{Chess} & \chesskeyone & \mascfirstline{simicschess60x60.dat.min} & \mascfirstline{simicschess60x60.dat.max}	& \mascfirstline{simicschess60x60.dat.std} & \mascfirstline{simicschess60x60.dat.avg} \\ %\cline{2-6}
      & \chesskeytwo & \mascfirstline{simicschess84x84.dat.min} & \mascfirstline{simicschess84x84.dat.max} & \mascfirstline{simicschess84x84.dat.std} & \mascfirstline{simicschess84x84.dat.avg} \\ %\cline{2-6}
      & \chesskeythree & \mascfirstline{simicschess118x118.dat.min} & \mascfirstline{simicschess118x118.dat.max} & \mascfirstline{simicschess118x118.dat.std} & \mascfirstline{simicschess118x118.dat.avg} \\ \hline
      \multirow{3}{*}{Julia} & \juliakeyone & \mascfirstline{simicsjulia225.dat.min} & \mascfirstline{simicsjulia225.dat.max} & \mascfirstline{simicsjulia225.dat.std} & \mascfirstline{simicsjulia225.dat.avg} \\ %\cline{2-6}
      & \juliakeytwo & \mascfirstline{simicsjulia450.dat.min} & \mascfirstline{simicsjulia450.dat.max} & \mascfirstline{simicsjulia450.dat.std} & \mascfirstline{simicsjulia450.dat.avg} \\ %\cline{2-6}
      & \juliakeythree & \mascfirstline{simicsjulia900.dat.min} & \mascfirstline{simicsjulia900.dat.max} & \mascfirstline{simicsjulia900.dat.std} & \mascfirstline{simicsjulia900.dat.avg} \\ \hline
    \end{tabular}
    \vspace{5pt}
    \caption{Software rasterization results in Simics.}
    \label{tab:keyvalsimics}
  }
  \hfill
  \parbox{.5\textwidth}{
    % tab:keyvalpara
    \centering
    \begin{tabular}{|c|c|c|c|c|c|}
      \hline
      \multirow{2}{*}{Benchmark} & \multirow{2}{*}{Input} & \multicolumn{4}{p{4cm}|}{\centering Elapsed time (\milli\second )} \\
      \cline{3-6} && \multicolumn{1}{c|}{Min} & \multicolumn{1}{c|}{Max} & \multicolumn{1}{c|}{Std} & \multicolumn{1}{c|}{Avg} \\ \hline
      \multirow{3}{*}{Chess} & \chesskeyone & \mascfirstline{parachess60x60.dat.min} & \mascfirstline{parachess60x60.dat.max} & \mascfirstline{parachess60x60.dat.std} & \mascfirstline{parachess60x60.dat.avg} \\
      & \chesskeytwo & \mascfirstline{parachess84x84.dat.min} & \mascfirstline{parachess84x84.dat.max} & \mascfirstline{parachess84x84.dat.std} & \mascfirstline{parachess84x84.dat.avg} \\
      & \chesskeythree & \mascfirstline{parachess118x118.dat.min} & \mascfirstline{parachess118x118.dat.max} & \mascfirstline{parachess118x118.dat.std} & \mascfirstline{parachess118x118.dat.avg} \\ \hline
      \multirow{3}{*}{Julia} & \juliakeyone & \mascfirstline{parajulia225.dat.min} & \mascfirstline{parajulia225.dat.max}	& \mascfirstline{parajulia225.dat.std} & \mascfirstline{parajulia225.dat.avg} \\
      & \juliakeytwo & \mascfirstline{parajulia450.dat.min} & \mascfirstline{parajulia450.dat.max} & \mascfirstline{parajulia450.dat.std} & \mascfirstline{parajulia450.dat.avg} \\
      & \juliakeythree & \mascfirstline{parajulia900.dat.min} & \mascfirstline{parajulia900.dat.max} & \mascfirstline{parajulia900.dat.std} & \mascfirstline{parajulia900.dat.avg} \\ \hline
    \end{tabular}
    \vspace{5pt}
    \caption{Paravirtualization results in Simics.}
    \label{tab:keyvalpara}
  }
\end{table*}


% Benchmark: Chess
\paragraph{Benchmark: Chess}
\label{par:experimentalmethodology_benchmarking_benchmarkchess}
To stress the latency between target and host systems, the 'Chess' benchmark performs a multitude of lightweight OpenGL invocations per frame, rendering a grid of chess-like tiles.
For each frame rendered, depending on the number of tiles, the benchmark performs a large number of magic instructions.
This induces high utilization of the Simics Pipe, which is intended to stress suspected magic instruction overhead.

A long sequence of draw calls is representative of drawing multitudes of shapes with OpenGL, such as a UI.
Accordingly, the benchmark is suitable for the purpose of representing a large number of graphics invocations.

The Chess benchmark is run with $60\times60$, $84\times84$, and $118\times118$ tiles, which entails $9\times60\times60$, $9\times84\times84$, and $9\times118\times118$ magic instructions per frame.

% Benchmark: Julia
\paragraph{Benchmark: Julia}
\label{par:experimentalmethodology_benchmarking_benchmarkjulia}
To stress the computational prowess of paravirtualized graphics in Simics, the 'Julia' benchmark performs a lone, computationally intensive, OpenGL invocation that render the Julia fractal~\masccite{web:tsiombikas:2014}.
Like a benchmark that renders a grid of tiles, where the grid resolution may be adjusted, the computation of a fractal is trivially scalable in terms of complexity, by modifying the number of iterations the computing algorithm performs.
Therefore, the Julia benchmark is suitable to profile a computationally intensive workload.

The Julia benchmark is run with $225$, $450$, and $900$ iterations, all of which induce $16$ magic instructions per frame.

\subsection{Threats to Validity}
\label{sec:threatstovalidity}
Because of complications caused by virtual time, measuring time in system simulation sometimes dictate special measures.
For instance, in terms of real-time rendering, the observer is far more interested in a frame rate relative to wall-clock time rather than virtual time.

In order to measure frame time in relation to wall-clock time, profiling must take place outside of the simulation.
One way of achieving this is to listen in on activity passing through a target serial port; this is a traditional front-end to the machine.
In this way, a simulation breakpoint can be triggered at the occurrence of a certain sequence of bytes written to a UART serial port.
This is the method used to measure frame time in Simics.

When using serial ports in this manner, one may introduce a profiling cost.
For example, file descriptors do not immediately transmit a byte sequence over the system UART.
For our set-up, we measure this overhead cost to be, on average, $1.5$~\milli\second .
If not specified otherwise, presented results take into account this average.

\subsection{Results}
\label{sec:results}
Results accumulated from software rasterized and paravirtualized execution are presented in Tables~\ref{tab:keyvalsimics}~and~\ref{tab:keyvalpara}.
In Figure~\ref{fig:histogram}, the results are presented as histograms, visualizing elapsed time in milliseconds to sample density.
For each experiment, collected frame time samples ($1000$) are subdivided into $100$ bins.
Any measurements outside of the standard deviation are not included in the figures.

Figure~\ref{fig:histogram} indicates that the Chess benchmark, when software rasterized, yields a broad sample distribution, seemingly distributed around a single point.
The right-hand side of the graph, showing impaired performance induced by paravirtualization, visualize a distribution decrease.
This is supported by the data presented in Table~\ref{tab:keyvalpara}.
We observe that software rasterization outperforms its paravirtualized counterpart, regardless of the number of tiles rendered.
The benchmark is devised to identify any bottlenecks related to the number of paravirtualizaed invocations.
Evidently, the prediction of a target-to-host communication latency issue is confirmed.

In Simics, magic instructions incur a context switch cost when resuming execution on the host.
This causes the simulation to no longer execute natively, which inhibits performance imrovements granted by hardware-assisted virtualization.
It also entails that Simics can no longer utilize JIT, forcing the simulator to rely on interpretation.
As such, in great numbers, magic instructions may affect performance.

By measuring the elapsed time of using magic insructions to escape simulation $1000$ times, we conclude that $1000$ consecutive magic instructions induce an average overhead of $5$~\milli\second\ (not taking into account any profiling cost).
These findings indicate that magic instruction overhead could account for the majority of elapsed frame time.

Figure~\ref{fig:histogram} shows that the Julia benchmark yields double to triple peak sample density destribution, both in software rasterized and paravirtualed Simics.
What causes this behavior is unclear; the fractal algorithm should perform roughly equally frame-to-frame.
The benchmark is intented to demonstrate how paravirtualization performs under computational stress.
This is where the benefits of hardware acceleration should be made apperent.
Accordingly, weaknesses in software rasterization is highlighted with frame times above the two second mark; corresponding maximum paravirtualized frame time measuring a mere $156$ \milli\second.
This confirms performance improvements of using paravirtualization, paired with magic instructions, to accelerate graphics.

These measurements are collected using hardware-assisted virtualization for accelerated virtualization.
If hardware-assisted virtualization is not available, such as if the simulated platform is other than x86, we expect a major hit to performance.
For software rasterization, this impact accounts for frame time increases well over two orders of magnitude.
Meanwhile, performance impacts to paravirtualization is often not significant, sometimes as low as a third of the original frame time.
For the Chess benchmark, paravirualized frame time increase with \textit{up to} one order of magnitude, one order less than the performance hit to software rasterization.
Across the board, paravirtualization suffer less performance impact, rendering the benchmarks up to three orders of magnitude faster than software rasterization.
We conclude that the effects of paravirtualization increase by one order of magnitude without hardware-assisted virtualization.
This entails that workloads that are otherwise sub-optimal for paravirtualization -- those performing a large amount of function invocations -- bring about performance improvements when utilizing paravirtualization.
Thus, the impact of magic instruction overhead is reduced, likely because a costly context switch is not inflicted on Simics.
We reason that some software rasterized worklads (Chess) may attain decent simulation performance simply because of a fast simulator; when native execution is not available, neither JIT nor interpretation may attain the same speeds as paravirtualization.

The samples collected without hardware-assisted virtualization are not presented in detail, since the scenario of native execution is far more likely.


% fig:histogram
\begin{figure}
  \setlength{\abovecaptionskip}{0pt}
  \setlength{\belowcaptionskip}{0pt}
  \centering
  \input{gnuhistogramssimicsparachess.tex}
  \input{gnuhistogramssimicsparajulia.tex}
  \caption{Sample histograms depicting $1000$ frames subdivided into $100$ bins, presented in milliseconds. Top 2x3: Chess. Bottom 2x3: Julia.}
  \label{fig:histogram}
\end{figure}

% conclusion.tex

% Conclusion
\section{Conclusion}
\label{sec:conclusion}
In section \ref{sec:results}, we have established strengths and weaknesses of paravirtualized graphics in the Simics full-system simulator; most notably, the bottleneck introduced by magic instruction overhead.
As such, we have confirmed original suspicions through the use of our benchmarks.
Thus, our study has identified the performance bottleneck inherent in great numbers of paravirtualized function invocations for magic instructions.

Furthermore, compiled results have showcased great improvements for computationally intensive graphics, as demonstrated by the Julia fractal benchmark, compared to its software rasterized Simics counterpart.
As such, we have accelerated graphics by up to $34$ times, reducing frame time from that of $1415$~\milli\second\ to the real-time feasible count of 42~\milli\second .
For platforms that cannot utilize hardware-assisted virtualization, paravirtualization has been estimated to obtain an additional order of magnitude in frame time reduction in comparison to software rasterization.
Such scenarios may include simulating other platforms than x86, such as PowerPC, or when utilizing certain advanced features in Simics -- e.g. breakpoints.

Accordingly, our experiment has identified the potential of using paravirtualization with shared memory for the means of accelerating graphics to that of real-time performance, testimonial to the results presented by Lagar-Cavilla et al. in their work on using paravirtualization to accelerate graphics~\dvtcmdcitebib{inproceedings:lagarcavilla:2007}.
Additionally, beyond that of accelerated graphics, results indicate performance improvements in terms of maximum frame times, inducing significantly improved standard deviation.
In line with stable frame rates being prerequisites for real-time applications, this further indicates, in coagency with reduced frame times, the feasibility of utilizing paravirtualized methodology for the purposes of accelerating graphics within virtual platforms.

To summarize: this paper has presented a solution for graphics acceleration implemented in the Simics full-system simulator by the means of paravirtualization.
The end-result is a solution which may generate libraries imitating EGL and OpenGL libraries.
This solution may effectively spy on application EGL utilization, without inhibiting said exchange, allowing unmodified OpenGL applications to be accelerated from within the simulation target.
The implementation communicates by the means of low-latency magic instructions with no limit as to how much memory may be shared.
As such, throughout this document, we have tackled and presented several issues pertaining to paravirtualized graphics acceleration.
For the purposes of performance testing, we have developed benchmarks with the distinct purpose of highlighting solution weaknesses and strengths.
We have presented an analysis of benchmarking results and concluded the benefits and drawbacks of paravirtualization as means to graphics acceleration in virtual platforms, backed by hard data stressing key points in the implementation.
Accordingly, the findings of this paper has contributed to our understanding of the difficulties facing paravirtualized graphics acceleration, and has established the feasibility of using paravirtualization to accelerate graphics in virtual platforms to that of real-time qualities.

To conclude: This paper has demonstrated performance improvements by accelerating graphics using paravirtualization.
Induced benefits are performance improvements of up to $34$ times -- or two orders of magnitude, paired with larger benefits in non-hardware-assisted virtualized use-cases.
Magic instruction overhead has been identified as the main performance bottleneck.
As such, a drawback of graphics paravirtualization is a weakness to large amounts of framework invocations.
Thus, this paper claims paravirtualization as a successful formula for system simulator graphics acceleration, and suggests utilizing high-level paravirtualization to accelerate graphics in virtual platforms.

The presented implementation may be advanced in a number of ways in order to support a higher number of platforms and an array of performance enhancements.
Below, recommendations for future study are presented.

In terms of performance, command serialization batching should be considered for the purposes of minimizing the number of performed magic instructions.
That is, the ability to queue framework invocations and transmit them in a batch rather than individually.
Considering that magic instructions are a performance bottleneck, such an optimization could drastically improve simulation performance.

Often, it is desirable to simulate systems other than the simulation host.
One could pose the scenario of a Linux host system simulating a machine running a Windows OS.
In this case, it is possible that target software utilize the DirectX framework to render graphics, whereas the host Linux system only feature the OpenGL libraries.

In $2014$, Valve released software capable of converting DirectX~$9.0$c-code to that of OpenGL~\dvtcmdciteref{technicaldocs:valve:2014}.
Albeit limited in its capabilities, the functionality of translating between one, platform-specific, framework to that of a cross-platform one may be practical for the purposes of graphics acceleration in virtual platforms.
An inherent flaw in paravirtualizing graphics APIs is that a target application may only make use of accelerated simulation performance if that application is implemented using the API in question.
If such a solution could be used to translate the rendering of a target program on-the-fly in a virtual machine, this might bridge the gap between the performance gains of using paravirtualization and other -- albeit slower -- more general ways of modeling graphics for any type of system.
Thus, the adaption of this sort of software could extend the capabilities and area-of-application of graphics paravirtualization in the Simics full-system simulator.

% Other possible future work to consider:
% * More verbose benchmarks.
% * Memory bandwidth benchmark (see EOF).
% * Consider mentioning making further study into how the overhead cost
%   for magic instructions might change if using VMP, JIT, or regular
%   interpretation.

%% Furthermore, and for the purposes of complementing this dissertation in particular, the author would like to suggest additional tests stressing target -to-host communications.
%% Preferably, said tests would stress the communication by other means than profiling the sampling of a large texture, since such a test may cause volatile performance in the reference material (being software rasterized Simics ), possibly due to cache misses (see section \ref{sec:threatstovalidity_benchmarkvariations}).
%% When performing such a test, it may be of value to profile the overhead induced by the memory table traversal described in section \ref{sec:proposedsolutionandimplementation_pagetabletraversal}.


% For peer review papers, you can put extra information on the cover
% page as needed:
% \ifCLASSOPTIONpeerreview
% \begin{center} \bfseries EDICS Category: 3-BBND \end{center}
% \fi
%
% For peerreview papers, this IEEEtran command inserts a page break and
% creates the second title. It will be ignored for other modes.
\IEEEpeerreviewmaketitle

\bibliographystyle{IEEEtran}
\bibliography{simics_opengl}

\end{document}

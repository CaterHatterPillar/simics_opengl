\section{Conclusion}
\label{sec:conclusion}
This paper presents graphics acceleration by the means of paravirtualization in the Simics full-system simulator.
The implementation generates EGL and OpenGL libraries and communicates with low-latency magic instructions.
To evaluate the implementation, benchmarks are developed to highlight solution weaknesses and strengths; an analysis of results is presented, as well as benefits and drawbacks of paravirtualized methodology.

In Section~\ref{sec:results}, compiled results showcase great improvements for computationally intensive graphics.
Performance is improved by up to $34$ times, reducing frame time from $1415$~\milli\second\ to $42$~\milli\second ; a frame time of $42$~\milli\second\ roughly corresponds to 24 FPS.
Furthermore, paravirtualization lower maximum frame times, significantly improving standard deviation.
Compared to software rasterization, paravirtualization is estimated to obtain an additional order of magnitude faster frame time without hardware-assisted virtualization.

Along with performance improvements, Section~\ref{sec:results} detail a communications latency issue inherent in magic instruction overhead.
This emerges as a performance bottleneck inherent in great numbers of paravirtualized function invocations.
While magic instructions are -- evidently -- fast enough to accommodate real-time graphics, improvements to this overhead should greatly improve their capacity.

To conclude: this paper demonstrates accelerated graphics in Simics using paravirtualization with magic instructions as a communications bridge.
Graphics acceleration in Simics is relevant because it facilitates debugging, testing, and profiling of software that depends on GPU utilization.
Consequently, paravirtualization is practical because it offers a good trade-off between development cost and performance.
The findings of this paper may help in extending the use of Simics to application development dependent on GPUs, including computer graphics and general-purpose workloads.

The benefits of paravirtualized graphics are performance improvements of up to two orders of magnitude, paired with larger benefits in non-hardware accelerated use-cases.
Magic instruction overhead is identified as the main performance bottleneck, resulting in a weakness to large amounts of framework invocations.
Accordingly, the findings of this paper contribute to our understanding of the difficulties facing graphics acceleration in virtual platforms by demonstrating the use of paravirtualization as a successful formula to accelerate graphics in Simics.

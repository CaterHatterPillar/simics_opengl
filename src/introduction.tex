% introduction.tex

% Introduction
\section{Introduction}
\label{sec:introduction}
Virtual platforms are becoming an important tool in the software industry in order to provide cost-effective time-to-market gains and meet the ever-shortening product life-cycles~\dvtcmdcitebib[p.~50,~52]{journals:magnusson:2002}\dvtcmdcitebib[p.~268]{journals:yi:2006}.
%(see \dvtcmdcitebib{magazines:magnusson:2004} for an overview of the benefits of full-system simulation)
Virtual platforms deliver these time-to-market -benefits in two major ways.
Virtual platforms enable pre-silicon development, that is; software development can begin prior to next-generation hardware being available~\dvtcmdcitebib[p.~52]{journals:magnusson:2002}.
Furthermore, virtual platforms may provide additional development tools compared to working with actual hardware.
For example, some virtual platforms allow simulated systems, often known as simulation targets, to be stopped synchronously without affecting timing or states of the target software~\dvtcmdcitebib[p.~61]{inproceedings:yu:2012}, and allow investigation of race conditions and other parallel programming issues~\dvtcmdcitebib[p.~1]{inproceedings:schumacher:2010}\dvtcmdcitebib[p.~7]{publications:leupers:2010}.
Additionally, such platforms may allow intricate inspection of simulated hardware, such as memory, caches, and registers~\dvtcmdcitebib[p.~54]{journals:magnusson:2002}.
Some virtual platforms provide advanced features such as reverse execution ~\dvtcmdcitebib[p.~30,~31]{publications:leupers:2010} (the ability to run a simulation backwards) and checkpointing ~\dvtcmdcitebib[p.~28,~29]{publications:leupers:2010} (functionality to save- and restore the state of a simulation).
These features are useful for debugging and testing a diverse range of software; from firmware to end-user applications~\dvtcmdcitebib[p.~25]{publications:leupers:2010}.

There are several techniques to provide fast functional virtual platforms that are running CPU-bound workloads.
Typical methods include interpretation ~\dvtcmdcitebib[p.~35]{journals:smith:2005}, just-in-time compilation~\dvtcmdcitebib[p.~24,~25]{journals:aarno:2013}, and hardware-assisted virtualization~\dvtcmdcitebib[p.~24,~25]{journals:aarno:2013}\dvtcmdcitebib[p.~38,~39]{publications:leupers:2010}.
Virtual platforms using these techniques can typically achieve a simulation performance in the range of $10$-$1000$ MIPS~\dvtcmdcitebib[p.~24,~25]{journals:aarno:2013}, but there is recorded performance of up to $5000$ MIPS~\dvtcmdcitebib[p.~38,~39]{publications:leupers:2010}.

The GPU has become a vital part in delivering good user experiences on many devices ranging from wearable, hand held, and portable units to desktop computers.
Since GPUs operate significantly different from CPUs, utilizing massively parallelized instruction sets to increase throughput, they continue to pose unique challenges to designers and developers~\dvtcmdcitebib[ch.~13]{publications:kirk:2010}.
The increased utilization of GPUs for general purpose workloads has extended the need for virtualization of such hardware in situations when hardware is busy, unavailable, not sufficient, or for the purposes of debugging and profiling~\dvtcmdciteref{web:microsoft:2013:warp}.
Yet seemingly few virtual platforms support virtualization of GPUs, despite their influence on modern computing.

For the purposes of simulation, different solutions follow varying use-cases.
For example, developers interested in benchmarking or driver development for next generation GPU- or CPUs may require detailed simulators that provide insight into execution engines and pipelines~\dvtcmdcitebib[p.~1]{inproceedings:schumacher:2010}.
Albeit applicable in certain use-cases and capable of running 'toy' applications, such platforms are often orders of magnitude too slow to run commercial workloads~\dvtcmdcitebib[p.~50]{journals:magnusson:2002}.
Application developers, on the other hand, do not necessarily care for the internal workings of hardware as they typically work at a higher abstraction level, for example; utilizing a graphics API that, in turn, communicate with the device driver.
As such, application developers may be more interested in achieving decent simulation performance rather than a timing-accurate processor model (see \dvtcmdcitebib[p.~30]{publications:leupers:2010} for an analysis of compromises in system simulation).
However, due to large differences in-between CPU and GPU architecture, simply delegating GPU-bound workloads to the CPU is rarely feasible in terms of performance.

Common approaches to achieve better simulation performance includes creating a functionally accurate model of the GPU, where internal details may be simplified, or using software rasterization without involving the GPU model.
However, these methodologies traditionally incur heavy performance losses in comparison to hardware acceleration.
In order to achieve better performance, one may 'offload' such kernels to the GPU of the system on which the simulation runs - often referred to as the simulation host.
There are a number of ways to do so, such as relying on PCI~passthrough and similar technologies to grant access to the underlying host hardware from within the virtual platform~\dvtcmdcitebib[p.~415,~416]{inproceedings:regola:2010}\dvtcmdciteref{web:jones:2009}, or utilizing a concept commonly referred to as 'Paravirtualization' at a higher level of abstraction (e.g., the OpenGL library).

Paravirtualization is a relatively new term defined as selectively modifying the virtual architecture to enhance scalability, performance, or simplicity~\dvtcmdcitebib[p.~165-166]{magazines:bartholomew:2006}.
Effectively, this entails modifying the virtual machine to be similar, but not identical, to host hardware~\dvtcmdcitebib[p.~165]{journals:barham:2003}.
As such, one may simplify the virtualization process by neglecting some hardware compatibility~\dvtcmdcitebib[p.~1]{inproceedings:youseff:2006}.

Implementing GPU simulation by the means of Paravirtualization provides the benefits of improved simulation performance, albeit it may entail loosing some of the benefits a virtual platform can provide.
For example, it may be challenging to support advanced features such as deterministic execution, checkpointing, and reverse execution.
However, paravirtualization entails modifying the simulated system, as some part of the target machine must be modified to accommodate the paravirtualized technology.

This paper comprises an investigation of paravirtualization of OpenGL~ES in the Simics virtual platform.
The work presents an implementation of accelerated OpenGL~ES~$2.0$ graphics by the means of paravirtualization; using magic instructions as a communications bridge.
In addition to this, we present benchmarks stressing important attributes of the devised solution.
Using the results collected from the execution of said benchmarks, on a number of platforms, the solution is evaluated.
Furthermore, the study identifies performance bottlenecks obstructing large numbers of paravirtualized invocations for the purposes of real-time graphics.
The results presented in this paper show performance improvements of up to $34$ times compared to software rasterized counterparts.

% TODO: The final paragraph of the paper should _briefly_ present the following sections and the layout of the paper. Complement when the contents are pretty much clear.

\hl{The remainder of the paper present the objectives and question formulations adhering to the presented study.
Additionally, an introductory chapter on relevant background and preceding work in the area - expanded upon by a succeeding chapter concerning the Simics full-system simulator - means to bring about many of the terms and concepts used in the subsequent material.
Next, a description of the devised paravirtualized implementation is presented along with a chapter on experimental methodology.
Furthermore, this paper features an elaboration on potential threats to validity before detailing the results collected using the methods described in prior chapters.
The material is then finalized by a speculative discussion chapter, analyzing the paravirtualized solution in regard to advanced functionality in the Simics full-system simulator, foregoing the conclusions which may be established from the compiled material.
Finally, the paper is concluded by a report on recommended future work in the area.}

% problemformulation.tex

% Problem Formulation
\section{Problem Formulation}
\label{sec:problemformulation}
The study presented in this document consists of implementing paravirtualization of a graphics API (being OpenGL~ES~$2.0$ ) in the Simics full-system simulator developed by Intel\circledR and sold through Intel\circledR 's subsidiary Wind River Systems, Inc. 
The solution devised to accelerate OpenGL adheres to that of the Android emulator, which is elaborated upon in section \ref{sec:backgroundandrelatedwork_qemu}.
As such, this study concerns investigating the performance, and the feasibility of extended benefits and advanced functionality, of paravirtualized graphics in a virtual platform.
This entails investigation, analysis, and development of methods and techniques for efficient communication and execution in the Simics run-time environment.
Furthermore, the study comprises analysis of the liabilities of paravirtualized technologies in regards to Simics philosophy (being high-performance determinism and repeatability~\dvtcmdcitebib{journals:aarno:2013}).
Thus, the study does not exclusively concern Simics integration, but an investigation of paravirtualized libraries in virtual platforms.

For the purpose of this paper, a paravirtualized solution for graphics acceleration in Simics is developed.
Furthermore, to accommodate the analysis of benefits and drawbacks of paravirtualized graphics, a number of benchmarks are devised to stress key points in the developed solution; with the goal of locating performance bottlenecks.
The benchmarks are designed to stress latency and bandwidth in target -to-host communication, in addition to computational intensity brought on by complex GPU -bound workloads.

Based on the devised solution, in coagency with the benchmarks, this study comprises an analysis of the performance of paravirtualized graphics compared to that of traditional software rasterization.
The objectives of the paper is to evaluate the feasibility of paravirtualization as an approach to accelerate graphics in virtual platforms; along with identifying its strengths and weaknesses.

In line with previous work in the area specified in chapter \ref{cha:backgroundandrelatedwork}, there has been no indication - in academic writing - of any pre-existing solution of paravirtualized graphics APIs signifying deterministic behavior; paving the way for supporting reverse execution graphics.
Such functionality could simplify debugging, testing, and profiling of applications comprising some GPU -bound workload; not limited to graphics- or GPU utilization in its entirety.
Entailed by these research gaps, the research questions formulated in this chapter are considered to be lacking in the field.

As such, the study performed for the purposes of this paper is relevant to the field of computer science by expanding upon the the knowledge of graphics acceleration in virtual platforms; in terms of facilitating debugging, testing, and profiling of software dependent on GPU graphics acceleration.
By the means outlined in chapter \ref{cha:aimsandobjectives}, this paper contributes to the field of computer science by answering these questions from the perspective of graphics paravirtualization in the Simics full-system simulator.
Accordingly, the key investigatory attributes and explicit research question formulations, sought to be answered by this paper, are presented below.

\newcommand*\researchquestionitem[2]{\item[#1:] \textit{#2}}
\begin{itemize}[noitemsep]
	\researchquestionitem{1}{What constitutes a viable implementation of paravirtualized graphics?}
	\researchquestionitem{2}{What are the benefits and disadvantages of paravirtualized graphics?}
	\researchquestionitem{3}{How does paravirtualized graphics performance relate to software rasterization?}
\end{itemize}

Furthermore, the study pertains to a number of discussion topics in line with the solution's relation to the Simics full-system simulator. These discussion questions presented below.

\begin{itemize}[noitemsep]
	\researchquestionitem{1}{What inhibits deterministic execution of paravirtualized graphics?}
	\researchquestionitem{2}{What inhibits checkpointing of paravirtualized graphics?}
	\researchquestionitem{3}{What inhibits reverse execution of paravirtualized graphics?}
\end{itemize}

% problemformulation.tex

% Problem Formulation
\section{Problem Formulation}
\label{sec:problemformulation}
This paper presents the paravirtualization of a graphics API (being OpenGL~ES~$2.0$) in the Simics full-system simulator developed by Intel\circledR\ and sold through Intel\circledR 's subsidiary Wind River Systems,~Inc. 
As such, this study concerns investigating the performance of paravirtualized graphics in a modern virtual platform.
This entails investigation, analysis, and development of methods and techniques for efficient communication and execution in the Simics run-time environment.

Accordingly, we devise a paravirtualized solution for graphics acceleration in Simics.
To accommodate the analysis of benefits and drawbacks of paravirtualized graphics, we devise a benchmark suite; with the goal of locating performance bottlenecks.
The benchmarks are designed to stress latency in target-to-host communication, in addition to computational intensity brought on by complex GPU workloads.
In coagency with the benchmarks, this study comprises an analysis of the performance of paravirtualized graphics compared to that of traditional software rasterization.
The objectives of the paper is to evaluate the feasibility of paravirtualization as an approach to accelerate graphics in virtual platforms; along with identifying its strengths and weaknesses.

\hl{In regards to previous work in the area, there is no indication - in academic writing - of existing paravirtualized graphics in a simulator with advanced capabilities such as Simics; featuring deterministic execution, checkpointing and reverse execution.
Potential performance gains on such a platform are inherently unclear due to these features.}
Such functionality could simplify debugging, testing, and profiling of applications comprising some GPU-bound workload.
Entailed by these research gaps, the research questions formulated in this chapter are considered to be lacking in the field.

As such, the study performed for the purposes of this paper is relevant to the field of computer science by expanding upon the the knowledge of graphics acceleration in virtual platforms; in terms of facilitating debugging, testing, and profiling of software dependent on GPU graphics acceleration.
By these means, this paper contributes to the field of computer science by answering these questions from the perspective of graphics paravirtualization in the Simics full-system simulator.
Accordingly, the research questions sought to be answered by this paper are presented below.

\begin{enumerate}
  \item What constitutes a viable implementation of paravirtualized graphics?
  \item What are the benefits and disadvantages of paravirtualized graphics?
  \item How does paravirtualized graphics performance relate to software rasterization?
\end{enumerate}
